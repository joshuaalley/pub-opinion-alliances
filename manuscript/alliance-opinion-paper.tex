\documentclass[12pt]{article}

\usepackage{fullpage}
\usepackage{graphicx, rotating, booktabs} 
\usepackage{times} 
\usepackage{natbib} 
\usepackage{indentfirst} 
\usepackage{setspace}
\usepackage{grffile} 
\usepackage{hyperref}
\usepackage{adjustbox}
\usepackage{amsmath}
\usepackage{siunitx}
\usepackage{multirow}
\setcitestyle{aysep{}}


\singlespace
\title{\textbf{Public Attitudes Towards Military Alliances}}
\author{Joshua Alley \\
Postdoctoral Research Associate \\
University of Virginia.\thanks{Thanks to Erik Lin-Greenberg, Philip Potter, Justin Schon and Todd Sechser, as well as participants in the Democratic Statecraft Lab Research incubator, the Lansing B. Lee/Bankard Seminar in Global Politics, 2020 Annual Meeting of the Peace Science Society and 2021 Meeting of the International Studies Association for helpful comments. This project was reviewed by the University of Virginia IRB (Protocol 3866) and preregistration files for this study are hosted in an OSF repository at https://osf.io/g28zs.} \\
jkalley@virginia.edu
}
\date{\today}

\bibliographystyle{apsr}

\begin{document}

\maketitle 

\doublespace 

\begin{abstract}
Why do Americans support or oppose military alliances? 
Existing evidence on this question cannot determine whether elites lead or follow public opinion. 
In this article, I demarcate the extent of elite influence on alliance attitudes by showing how partisanship and foreign policy dispositions shape individual responses to elite cues and alliance characteristics. 
I then use two conjoint survey experiments to examine the roots of public attitudes towards forming or maintaining international alliances.  
I find that elite cues exert substantial influence, but important subsets of both parties hold strong attitudes and are unresponsive to elite cues. 
There is a partisan asymmetry in the content of these fixed attitudes, as staunch alliance supporters in the Democratic party ignore elite cues, while consistent alliance skeptics in the Republican party do not heed elites.  
The results imply that elites influence public opinion towards military alliances, but foreign policy dispositions constrain their impact.  
\end{abstract}


\newpage 


\section{Introduction}

% lay out the question
What determines U.S. public opinion towards military alliances? 
Despite the importance of public attitudes for forming and upholding U.S. alliances, we do not know why the public supports or opposes alliance commitments. 
Most existing evidence on this question comes from opinion polls measuring public sentiment towards salient alliances like the North Atlantic Treaty Organization (NATO).
These polls provide important descriptive data, but they do not explain why individuals express particular opinions. 


% puzzle
Standard polling data has limited explanatory power because alliance attitudes are subject to the longstanding puzzle of who leads whom in public opinion on foreign policy.
On the one hand, elites have ample opportunity to lead public opinion on alliances, given limited public information and interest in foreign policy \citep{Canes-Wrone2006, Druckman2014}.
On the other, leaders often follow public attitudes \citep{Barberaetal2019, HagerHilbig2020}.
In the context of alliance attitudes, it is unclear if elites lead the public or follow existing partisan differences in foreign policy dispositions that give Republicans and Democrats different  \footnote{This article considers the leading or following question for Trump and NATO: \url{https://fivethirtyeight.com/features/is-trump-fueling-republicans-concerns-about-nato-or-echoing-them/}}
Although the public pays little attention to international affairs, foreign policy dispositions give structure to opinions that elites could follow \citep{Holsti1992, PageShapiro1992, KertzerZeitzoff2017}.
To give an example from U.S. politics, Republicans have an established history of isolationism and skepticism towards alliances. 


% contribution
In this paper, I explore the roots of alliance attitudes by unpacking the extent and limits of elite influence. 
To do this, I examine how foreign policy dispositions and partisanship change the way individuals respond to elite cues and alliance characteristics.
Whether elites lead or pander to public opinion in the Republican and Democratic parties depends on how elite cues impact individuals with different predispositions towards alliances from isolationism and hawkishness.  
Isolationists are more likely to oppose alliances, while hawks are inclined to back alliance participation. 
If elite cues encourage isolationists to support alliances and hawks to oppose alliances, they clearly lead alliance attitudes. 
I find that while most individuals in both parties follow co-partisan elite cues regardless of their foreign policy dispositions, the strongest alliance supporters in the Democratic party hold rigid alliance attitudes, as do staunch alliance skeptics in the Republican party. 
Most alliance attitudes are plastic under elite cues, but some individuals hold rigid opinions. 


% Importance part 1: public opinion undergirds alliance com in democ
In addition to providing new insight into the puzzle of who leads whom, there are three reasons that understanding U.S. public opinion towards alliances is worthwhile. 
To start, public opinion is central to debates over whether democracies make more reliable commitments.\footnote{Public opinion is important, but it is not deterministic. \citet{Kreps2010} notes that public disapproval may not hinder coalition warfare, especially when elite consensus favors fighting.}
NATO leaders often feared that changing public attitudes would undermine the alliance \citep{Sayle2019}.   
If public opinion towards alliances is indifferent to elite cues, stable attitudes and reliable commitments follow \citep{Gaubatz1996}.
If elite cues drive public opinion, then elite alliance skeptics can bring the public along, leading to cycles that hinder democratic reliability \citep{GartzkeGleditsch2004}.
For example, some observers feared that Donald Trump's rhetoric would undermine domestic support for alliances.
Yet U.S. public approval of alliances like NATO remained steady during the Trump administration \citep{PewNATO2020}. 


% importance part 2: practical relevance- US role in world. 
Why the public supports or opposes alliances also speaks to the consequences of a prominent scholarly and policy debate. 
Two competing visions of U.S. foreign policy depend heavily on alliances. 
One view believes that the United States should reduce its alliance commitments to pursue a restrained grand strategy \citep{Preble2009, Posen2014}.
The other argues that continued deep engagement through alliances is the best way to promote U.S. security and prosperity \citep{Brooksetal2013, BrandsFeaver2017}. 
If elite cues drive public opinion, leaders will be free to implement their preferred vision. 



% importance part 3: WHY support international cooperation, not just consequences of international institutions
In addition to its practical importance, this study fills a gap in international institutions scholarship. 
Scholars are more likely to study how international institutions affect public attitudes (e.g. \citep{KayaWalker2014, Greenhill2020}), than scrutinize the sources of public attitudes towards international institutions themselves. 
Other studies use observational survey data to examine public opinion towards international cooperation such as multilateral financial institutions \citep{Edwards2009} or the United Nations \citep{Torgler2008, DellmuthTallberg2015}. 
That leaves limited causal evidence on why individuals hold particular alliance attitudes.
In one study of public opinion and military alliances, \citet{TomzWeeks2021} show that the presence of an alliance increases public support for foreign military intervention. 
\citet{Chuetal2021} explore how values and interest based elite cues shape public attitudes towards alliance maintenance. 
I build on these works with more general experiments on alliance formation and maintenance. 


% Assess w/ a survey experiment
To provide causal evidence on the sources of public opinion towards alliances, I use two conjoint survey experiments.
Conjoint experiments randomize multiple alliance characteristics and elite cues, so this tool is well-suited to assessing the relative weight of different factors \citep{Hainmuelleretal2014}.
Unlike in observational data, in the context of an experiment that randomly assigns elite cues, I can use information on foreign policy dispositions within parties to distinguish who leads and who follows. 
The first study asks individuals to rate five hypothetical new alliance commitments and support or oppose alliance formation.
The second asks respondents to rate five hypothetical existing commitments and support or oppose alliance maintenance. 


% how address puzzle
In my analysis, I show who follows elite cues. 
I start by estimating the unconditional average marginal component effects of different elite cues and alliance attributes.
Then I examine how partisanship and foreign policy dispositions shape individual responses to elite cues and alliance characteristics. 


% findings
In nationally representative survey experiments on alliance formation and maintenance, I find that alliance attitudes depend on elite cues, partisanship and foreign policy dispositions. 
Elite cues are influential, but their impact depends on foreign policy dispositions and partisanship, which also shape baseline support for alliances. 
Hawkish individuals in both parties often support alliance participation, even if they also hold isolationist views that otherwise dampen alliance support.
There are also salient partisan differences in the role of foreign policy dispositions. 


Some Democrats and Republicans hold such strong alliance attitudes that they are unresponsive to elite cues.
Hawkish and isolationist Democrats are the strongest alliance supporters.
Dovish and isolationist Republicans are the greatest alliance skeptics. 
Other partisans shift their alliance attitudes in response to co-partisan elite cues. 
Therefore, Republicans can lead the most likely alliance supporters in their party, while Democrats can lead relative alliance skeptics. 
If elected leaders follow the strongest alliance attitudes in their party, they will lead in competing directions. 


% partisan differences in democracy, region
Some alliance characteristics also impact public attitudes. 
Democrats and most Republicans prefer alliances with other democracies to supporting nondemocracies. 
Issue linkages also increase public support for alliance participation, while high financial costs reduce support for alliance maintenance. 
Some Republicans express strong regional preferences and express minimal interest in alliances with African states. 


% differences between formation and mainteance
Last, I find that public support for alliance formation and maintenance command differs.
Even with elite opposition, upholding existing alliances almost always retains majority support on average. 
In alliance formation, elite cues determine whether a new treaty has majority or minority support. 
Therefore, elites have more influence over adding new obligations than changing existing commitments. 


% Implications
The results imply that elites have substantial influence on public attitudes towards alliances. 
Foreign policy dispositions make some individual opinions less plastic, however. 
Elites often lead public opinion towards alliances, but subsets of both major parties hold rigid attitudes.


These findings speak directly to the future of U.S. alliance politics. 
Although cues from one type of elite can weaken public support for U.S. alliances, they cannot move attitudes towards existing treaties into majority opposition on their own, given high baseline support for alliance maintenance.
Bipartisan opposition to alliances would command enough support for leaders to withdraw from an alliance with little public disapproval.  
Public support for new alliance commitments is especially vulnerable to elite criticism. 

% Try it w/o the plan of paper for flow. 


\section{Who Leads Alliance Attitudes?}


Public opinion has a critical role in democratic foreign policy and alliance politics.
Public approval guides elite military intervention decisions \citep{Tomzetal2020, LinGreenberg2021}. 
In democracies, anticipation of public audience costs of treaty violation encourages limited promises of military support \citep{Chibaetal2015, FjelstulReiter2019}. 
Moreover, public attitudes have a key role in disputes about the reliability of democratic alliances \citep{Gaubatz1996, GartzkeGleditsch2004}. 
Last, policymakers pay careful attention to public support for alliances \citep{Sayle2019}. 


% NATO example to bring out the puzzle
There is also meaningful variation in pubic opinion towards military alliances. 
\autoref{fig:nato-op-time} plots the percentage of respondents supporting NATO in 59 surveys from 1974 to 2020.\footnote{These surveys ask respondents to assess NATO in many ways. I consider favorable opinions, feeling thermometer ratings of 50 or higher, and support for increasing or maintaining U.S. commitment as indicators of support for NATO.} 
A majority of respondents back NATO in most surveys, but public support has fallen since 2000.  


\begin{figure}
	\centering
		\includegraphics[width=0.95\textwidth]{../figures/nato-op-time.png}
	\caption{US public support for NATO from 1974 to 2020. Each point marks a unique poll, and colors differentiate the percentages of respondents that expressed support, opposition or neutral/no opinion of NATO. Loess lines estimate the average support for each group in every year. Topline data from the Roper Center's iPoll database.}
	\label{fig:nato-op-time}
\end{figure}



Observed alliance attitudes like those in \autoref{fig:nato-op-time} are subject to a longstanding puzzle in public opinion on foreign policy--- who leads whom? 
In this case, how much does public support for NATO depend on elite cues? 
Put differently, it is unclear if public attitudes towards alliances are a top-down result of elite cues or if established public attitudes drive elite cues. 
Both perspectives offer plausible models. 


Evidence on whether elites pander to or lead public opinion is divided.
Some scholars suggest that elites are more likely to lead public opinion. 
\citet{Canes-Wrone2006} finds that U.S. Presidents rarely follow the public if they disagree with public preferences.
\citet{JacobsShapiro2000} argue that elites are more likely to manipulate public opinion than follow it. 
\citet{Kreps2010} notes that public disapproval did not constrain participation in NATO's International Security Assistance Force in Afghanistan. 
Moreover, foreign policy is a secondary concern for many voters, so elite foreign policy views and rhetoric can diverge from public attitudes with few political repercussions \citep{BusbyMonten2012}. 


Other studies suggest that elites match their rhetoric and policy stances to public opinion. 
\citet{Barberaetal2019} use social media data to show that legislators are more likely to follow than lead public opinion on issues. 
\citet{HagerHilbig2020} find that exposure to public opinion research moves speech and policy positions by German politicians closer to majority opinion. 
\citet{GuisingerSaunders2017} note than for issues with low partisan polarization, information effects can dominate public opinion, though elite cues matter more for polarized issues like cap and trade schemes. 
\citet{Haesebrouck2019} uncovers little evidence that European elites led their public to support military interventions in Libya and the Islamic State. 
\citet{Bechteletal2015} find that elite cues and frames led Swiss individuals, especially those with low knowledge, to reinforce their existing attitudes. 
Even military elites shape their recommendations in response to public opinion \citep{LinGreenberg2021}. 



Alliance attitudes are subject to this puzzle or who leads whom. 
On the one hand, limited public information about alliances could give elite cues substantial influence \citep{Druckman2001}. 
On the other, the public opinion towards alliances may depend on individual dispositions and intuitions about international affairs, which provide consistent heuristics even with limited information \citep{Herrmannetal2009, KertzerZeitzoff2017}.
Some combination of the two is possible. 
\citet{PageShapiro1992} note that public opinion is broadly consistent and rational, and changes in predictable ways in response to information from multiple sources, including elite cues. 
Perhaps some alliance attitudes are more plastic than others. 


Understanding alliance attitudes therefore provides insight into a fundamental debate about public opinion on foreign policy.  
In the following, I unpack how partisanship and foreign policy dispositions provide leverage to understand whether public opinion towards alliances is plastic or rigid in the face of elite cues.  
First, I explain how elite cues can lead public views of alliances. 


\subsection{Elite Cues} 
% Framing/elite leading
% Debate over leading/pandering

Elite cues are a plausible determinant of alliance attitudes. 
This framework holds that the public follows trusted elites in forming their opinion, so elite portrayals of alliances bolster or undermine public support.
Thus, public opinion towards alliances is endogenous to elite views \citep{Druckman2014}.


In an elite leading model, public opinion towards alliances permeates down from the top.
There is substantial evidence that elites influence public foreign policy attitudes \citep{BaumPotter2008}. 
The media often convey elite cues and frames.
Social media may further amplify elite influence \citep{BaumPotter2019}.   


Furthermore, information shortcomings make individuals more responsive to elite framing and cues \citep{Druckman2001, Peterson2017}.  
The public has limited information about foreign policy relative to domestic issues.
Furthermore, alliance politics usually have low foreign policy salience. 
Alliances are less salient than international conflict, which is the most common issue in studies of foreign policy opinions. 
Therefore, elite support or opposition could have a profound influence on alliance attitudes because individuals rely on trusted elites in an issue environment with little other information. 


% cue-giver matters 
Elite identity shapes how they influence public attitudes.
Elected officials, diplomats and military leaders all participate in alliance politics.
Elected leaders have substantial public reach.  
Cues from military leaders can shape public opinion about the use of force \citep{Golbyetal2018}, so military endorsements may be especially influential. 
Diplomatic elites are also domain experts. 
Public perceptions that military leaders and diplomats are well-informed about alliances will likely increase their influence. 


% highlight partisanship 
Partisanship also shapes elite identity and influence.
Under partisan polarization, individuals discount messages from out-partisan elites. 
Conversely, trust makes cues from co-partisan elites more influential \citep{Druckmanetal2013}. 


% summary 
According to the elite cues model, support for alliances by trusted elites should increase individual support for alliances, and opposition will reduce support.  
This makes unified elite cues influential.
\citet{Berinsky2007} finds that unified elite support for war leads to robust public support. 


% transition paragraph: scope of influence
Elite cues are a straightforward and compelling explanation of alliance attitudes.
Even information about alliance characteristics like allied democracy or military spending likely reaches the public through elite sources. 
As they receive elite messages, individuals also hold prior attachments and beliefs, however.
Partisanship and individual foreign policy dispositions could modify how rigid or plastic alliance attitudes are in the face of elite cues. 


\subsection{Individual Concerns}


% Overview para
Foreign policy dispositions and partisanship shape individual perceptions of the costs, benefits and value of international cooperation. 
This has two consequences for alliance attitudes. 
First, they establish individuals' baseline alliance support, or willingness to back alliances in general.\footnote{Another way to think of baseline support is an individual disposition to support a generic alliance, regardless of elite cues or alliance characteristics.} 
Second, individual concerns might change individual responses to elite cues and particular alliance characteristics. 


% foreign policy disposition
Many individuals have stable intuitions about international politics. 
These principles shape how individuals respond to foreign policy decisions, such as backing down from military intervention promises \citep{KertzerBrutger2016}. 
Militant assertiveness and internationalism are two key foreign policy dispositions \citep{Herrmannetal1999}.  


% internationalists more likely
% Define internationalism  
Internationalism reflects an inclination to work with other countries and contribute to international institutions. 
Internationalist respondents support American engagement in foreign affairs and will likely favor alliance commitments. 
Conversely, isolationists are skeptical of international institutions and cooperation \citep{Kertzer2013}. 
Isolationists dislike foreign involvement and prioritize domestic affairs. 
As a result, isolationists should be skeptical of alliances and especially opposed to alliances with high financial cost or substantial obligations. 


% militant assertiveness 
Militant assertiveness increases support for alliance participation. 
Hawkish individuals are more willing to use force to address international problems. 
Although alliances are a cooperative institution, they also aggregate military capability \citep{FordhamPoast2014}, and call on members to support one another in war.
Hawks will support capability aggregation through alliance commitments and are willing to bear the risks of foreign wars.  
Dovish individuals are skeptical of using military force in general, so they are less likely to support military alliances that promise to fight for foreign countries.  


% partisanship 
Militant assertiveness and internationalism are connected to partisanship, as they vary across and within the Democratic and Republican parties. 
Conservatives in the United States have a longstanding history of isolationist sentiment, for instance \citep{Kupchan2020}.
As a result, they are usually more skeptical of international engagements like alliances. 
Republicans also tend to be more hawkish as well, however \citep{Gries2014}. 


Party identification also connects elite cues and individual concerns by determining whose cues matter.
Individuals look to cues from trusted elites, and partisanship is a straightforward heuristic for who to trust. 
Because partisanship reflects foreign policy dispositions and connects to elite cues, it has a crucial role in alliance attitudes. 


% limits 
Understanding alliance attitudes requires careful attention to elite cues, partisanship and foreign policy dispositions. 
Although elite cues are likely influential, the extent of elite influence is unclear. 
This is especially true because partisanship is correlated with foreign policy dispositions like isolationism and militant assertiveness that shape alliance attitudes. 
Perhaps Republican leaders' opposition to alliances does not decrease Republican support for alliances, it simply reflects conservative isolationism, for example. 



% partisanship puzzle
%At the same time, because partisanship and foreign policy dispositions are correlated, disentangling them is essential. 
%In particular, any analysis must establish whether partisan differences in alliance attitudes are the result of party affiliation or foreign policy attitudes that are correlated with partisanship. 
%To do this, I will compare the alliance attitudes of Democrats and Republicans with similar foreign policy dispositions. 
%If Republicans and Democrats with similar foreign policy dispositions hold distinct alliance attitudes, this implies that partisanship has some influence. 


\subsection{Leading, Pandering and Individual Concerns}


% how distinguished
To assess who elites lead, I leverage the confluence of partisanship and foreign policy dispositions.
I examine how elite cues impact Democrats and Republicans with different initial dispositions towards alliances.
How individuals with different levels of militant assertiveness and isolationism respond to cues from copartisan elites reveals the extent of elite leadership. 


% explain 
For elites to lead, they must overcome individual predispositions to support or oppose alliance participation. 
If elites lead public opinion, co-partisan elite support will increase support for alliance participation even among isolationists. 
No effect of elite support and a null or negative effect of elite opposition among isolationists implies limited elite leadership. 
Similarly, if elite opposition reduces support among hawkish individuals who would otherwise support an alliance, elite cues lead public opinion towards alliances. 
But if hawkish individuals discount elite cues, there will be a null or positive impact of elite support on alliance attitudes. 


\begin{table}[hbt!]
\begin{center}
\begin{tabular}{lccc}
   Model  & Elite Cue & Isolationists & Hawks  \\
\hline
\multirow{2}{*}{Elite Lead} & Support   & Increase Support  &  Increase Support \\
                            & Oppose    & Decrease Support  &  Decrease Support \\

 \hline
\multirow{2}{*}{Elite Follow} & Support   & Null  & Null or Increase Support \\
                              & Oppose    & Null or Decrease Support   &  Null \\
\hline
\end{tabular}
\caption{Summary of results consistent with elite leading or following public opinion on military alliances. These predictions assume that isolationists are disposed to oppose alliances, while hawks are likely to support alliance participation.}
\label{tab:arg-sum}
\end{center} 
\end{table}


\autoref{tab:arg-sum} summarizes the implications of this argument.  
Clear elite leadership implies that public attitudes follow elite cues, even if cues conflict with their likely disposition towards alliances. 
Isolationism and hawkishness overlap, so their relative weight is also an important concern.


% no a priori about different combinations
There are four foreign policy dispositions within each party. 
Individuals may be isolationist and hawkish, internationalist and hawkish, isolationist and dovish, or internationalist and dovish.
While some existing research does not divide isolationists into hawks and doves and distinguishes between cooperative and militant internationalists \citep{Kertzeretal2014}, I divide isolationists by hawkishness to assess the net impact of competing dispositions.\footnote{To streamline discussion across the four categories, I do not use the terms cooperative and militant internationalism in the manuscript, though the concepts are equivalent.}  
Dovish isolationists are the most likely alliance skeptics, while hawkish internationalists are the most likely alliance supporters. 
I do not have strong priors about the relative strength of hawkishness and isolationism, however.
One effect could dominate the other, the two factors could offset, or they could interact in unexpected ways.


% dividing partisans
Dividing respondents into these groups provides clear leverage over who holds plastic or rigid alliance attitudes. 
This in turn allows me to identify the boundaries of elite leadership. 
While most of the results focus on elite cues, I also consider alliance characteristics. 
Besides indications of support, elites and media often convey other information about an alliance that changes individual attitudes.  



\subsection{Alliance Characteristics}
% some alliances are more attractive than others
% look to other results- democ, strength, etc. 
% economic interests


Military alliances take many forms, as states negotiate distinct treaties with diverse partners.
Allied capability, shared interests, and the nature of the alliance obligations are all plausible determinants of alliance attitudes.   
Limited public information on foreign policy issues may constrain the impact of alliance characteristics, however. 
Furthermore, most information about alliance characteristics is attached to elite cues in media reports \citep{BaumPotter2008}. 


% Capability
Allied capability is the first major alliance characteristic.
Greater allied military capability should generally increase the appeal of an alliance. 
All else equal, alliances with militarily capable states are more valuable \citep{Johnsonetal2015}. 
Inasmuch as the public understands that allies with substantial military capability have greater capacity to deter and fight, they will back alliances with more capable states. 


% Shared interests
Perceptions of shared interests with allies are another salient alliance characteristic. 
Threat, economic ties, recent military operations and democratic political institutions are all observable indicators of common interests. 
If individuals see minimal shared security threat, they will be less likely to support protecting foreign states.
In economics, protecting trade ties often motivates asymmetric alliances between large and small states \citep{Fordham2010}. 
As individuals value foreign trade and investment and see it as an indicator of common interests, they will approve of alliance commitments with trade partners. 
If the United States and a potential alliance partner or current ally recently participated in a common military operation, the public could believe that they share common concerns and interests. 
Joining U.S. operations also suggests that the other state will bear substantial costs to support the United States.\footnote{Some states participate in wartime coalitions to encourage closer ties with the United States \citep{GannonKent2020}.}


% Democracy 
Democratic citizens may also prefer alliances with other democracies. 
At a minimum, the democratic public rarely supports military strikes against other democracies \citep{TomzWeeks2013}. 
Individuals may believe that democracies should cooperate because they share common concerns and values. 
Shared values are a strong justification of alliance participation \citep{Chuetal2021}. 
Democratic political regimes thus offer a simple heuristic for a trustworthy and valuable alliance partner. 


% Alliance treaty design 
In addition to shared interests, alliance obligations may shape public support for alliances, especially alliance formation. 
There is immense variation in alliance treaty content \citep{Leedsetal2002}.
Potential alliance members must agree on whether they are willing to offer military support, conditions on that support, and how they will contribute to deterrence and war fighting \citep{Poast2019a}. 
These issues are reflected in alliance treaties, which stipulate military intervention, conditions on allied support, defense cooperation, and issue linkages, all of which could impact alliance attitudes. 


% conditionality, depth and issue linkages
If the public fears entrapment in foreign conflicts or reckless allies, they will prefer alliances with conditional obligations.
Most U.S. alliances restrict intervention to attacks on allies and conflicts in specific regions. 
Other alliances supplement the core promises of military support with peacetime defense cooperation \citep{Morrow1994, LeedsAnac2005} to coordinate policies and establish credible commitments.
Almost half of all military alliances have some defense cooperation \citep{Leedsetal2002}, and many U.S. alliances include formal organizations, bases, and military aid. 
The public may prefer more arms-length commitments, or back strong ties with allies. 
Last, formal and informal issue linkages often facilitate agreement and support credible alliance commitments \citep{Poast2012, Poast2013}. 
Perceptions that an alliance brings trade or foreign policy concessions could increase public approval, as the public perceives tangible benefits.  


% financial costs
Finally, military alliances affect U.S. defense spending \citep{AlleyFuhrmann2021}. 
While the public may accept alliance expenses, skeptics often highlight the financial costs of alliance commitments \citep{Posen2014}. 
Military spending on alliances has opportunity costs, as funds spent on the military cannot acquire other goods. 
Thus, more expensive commitments should command less public support, especially among isolationists. 

 
% how to think about characteristics
Accounting for these alliance characteristics avoids confounding elite cues and provides insight into what information about an alliance shifts public attitudes. 
Providing greater detail ensures the any impact of elite cues is not driven by inferred alliance characteristics.
It also mimics media presentations of elite cues that give other information about an alliance. 


% formation vs maintenance
Before discussing the research design, there are two important considerations. 
First, alliance formation and maintenance are distinct processes \citep{Snyder1997}. 
Therefore, I consider alliance formation and maintenance in separate survey experiments to assess whether the public views making a new alliance commitment and upholding an existing treaty differently. 


%-------------------------------------
% % cut over time for the moment
%% explanation over time
%The elite cues model can predict changes in alliance attitudes over time. 
%In this framework, shifting public attitudes reflect new elite cues. 
%Conversely, stable public opinion would indicate consistent elite views. 
%
%
%% change over time: generations
%Individual concerns could explain changes in public support for alliances over time through generational shifts. 
%As the experiences of each generation mold their foreign policy dispositions, public opinion of alliances will slowly shift over time. 
%Such gradual shifts in public opinion towards alliance commitments are perhaps consistent with some of the observed data in \autoref{fig:nato-op-time}. 
%
%
%% shared interests and cap as key explanations of changes over time. 
%Indicators of shared interests and changes in allied capability could explain temporal variation in alliance attitudes. 
%Threat perceptions and recent military operations are both salient contextual factors. 
%If public perceptions of common interests wane, support for alliances will fall. 
%For example, reduced allied capability could create beliefs that partners are not pulling their weight or contributing enough capability. 
%Democratic backsliding might also lead the public to question whether allies share U.S. interests and values. 
%
%% obligations are less dynamic
%Alliance obligations are less likely to explain changes in public opinion over time than individual dispositions or contextual factors like threat, trade and allied democracy. 
%Fixed treaty obligation cannot explain variation in alliance attitudes over time. 
%Treaty obligations could affect public support for treaty formation, however. 
%--------------------------------------------



% long-run cycles
Second, feedback between elite cues and public opinion is plausible in the long run. 
It is possible that public opinion shapes elite cues, which in turn alter public opinion. 
For example, elites could respond to established alliance skepticism by adopting rhetoric that encourages further opposition. 
Elites could also attempt to lead public opinion and bolster alliance support, however.
Such feedback takes time, and would be most obvious in the context of longstanding alliances.
If such a cycle exists, elite cues must influence alliance attitudes.
This analysis can therefore establish part of a potential feedback cycle, and identify who elites can lead.  
I now describe how I assess the sources of alliance attitudes. 



\section{Research Design}


% justify conjoint: 
I use two conjoint survey experiments to unpack U.S. public support for forming and maintaining alliances. 
Information about observed alliances includes bundles of elite support and alliance characteristics. 
Conjoint experiments allow researchers to decompose such composite phenomena and compare multidimensional treatments \citep{Hainmuelleretal2014}. 
For example, \citet{HainmuellerHopkins2015} use a conjoint experiment to identify what types of immigrants Americans favor. 
%\citet{BechtelScheve2013} assess how institutional design affects public approval for climate cooperation agreements with a conjoint experiment in four countries. 


% describe rating tasks
Both conjoint experiments ask individuals to rate and support participation in defensive military alliances with randomly generated profiles of alliance characteristics and elite cues. 
In the alliance formation experiment, I ask respondents to assess five hypothetical new alliances. 
The alliance maintenance experiment presents five hypothetical existing alliances.


To start, I measure key respondent characteristics, especially partisanship, hawkishness and internationalism.  
These measures provide the basis of subgroup analyses examining how individual concerns shape baseline support for alliance participation as well as individual responses to elite cues and alliance characteristics. 
After measuring key individual factors, I present respondents with a table of information about a hypothetical alliance with a randomly generated profile of elite cues and characteristics.
Once respondents read the table, I ask them to rate the alliance on a scale from 0 to 100 and express approval of alliance formation or maintenance with a yes/no question. 
I will then present four more randomly generated alliance profiles, so each respondent rates five hypothetical alliances in a single-profile conjoint design.%\footnote{A two profile design would ask respondents to choose between two alliances, each with a random set of characteristics.} 


% Add a table with conjoint attributes. 
Each alliance partner profile is randomly generated from the attributes in \autoref{tab:conjoint-vars}.
Every attribute has multiple potential values.
I produce the full alliance profile by randomly selecting one value from each attribute. 
The set of attributes and values captures theoretically interesting alliance characteristics and generates plausible profiles.\footnote{There are no restrictions on value combinations in the alliance profiles. I employ this uniform randomization because all of these alliance profiles are plausible. This also generates substantial variance in elite cues.} % De la cuesta et al discussion here as needed
I randomize attribute order at the respondent level, so the table of attributes is the consistent for each respondent. 
Drawing alliance profiles at random and providing multiple rating tasks in a conjoint experiment makes estimating the average marginal component effect (AMCE) for each alliance attribute straightforward \citep{Hainmuelleretal2014}. 


\begin{table}
\begin{adjustbox}{width = .99\textwidth}
\begin{tabular}{lc} 
\hline \\ 
\textbf{Attributes} & \textbf{Values} \\
\hline \\ 
Republican Senators & Support an alliance with this country. \\
                    & Oppose an alliance with this country. \\ 
                    
Democratic Senators & Support an alliance with this country. \\
                    & Oppose an alliance with this country. \\ 
                    
The Joint Chiefs of Staff & Support an alliance with this country. \\
                    & Oppose an alliance with this country. \\ 
                    
The Secretary of State & Supports an alliance with this country. \\
                    & Opposes an alliance with this country. \\ 
                    
Trade Ties          & The United States has minimal trade ties with this country. \\
                    & The United States has modest trade ties with this country. \\
                    & The United states has extensive trade ties with this country. \\ 
% modified from Tomz and Weeks 2013 APSR: https://web.stanford.edu/~tomz/pubs/TomzWeeks-2013-11-Appendix.pdf 
Partner Political Regime    & This country is not a democracy, and shows no sign of becoming a democracy. \\
                    & This country is a democracy, but shows signs that it may not remain a democracy. \\ % democ backsliding
                    & This country is a democracy, and shows every sign that it will remain a democracy. \\
                    
Partner Military Capability & 10,000 soldiers and spends 1\% of their GDP on the military. \\ % low
                    & 80,000 soldiers and spends 2\% of their GDP on the military. \\ % moderate
                    & 250,000 soldiers and spends 3\% of their GDP on the military. \\ % high 
                    
Shared Threat       & The United States and this country face minimal common threats. \\ 
                    & The United States and this country face modest common threats. \\
                    & The United States and this country face serious common threats. \\
                    
Recent Military Cooperation  & This country has not participated in recent U.S. military operations. \\ 
                    & This country recently supported U.S. airstrikes against terrorists. \\
                    & This country recently supported U.S. counterinsurgency operations. \\
                    & This country recently fought with the United States in a war. \\
                    
Financial Cost      & This alliance requires \$5 billion in annual U.S. defense spending.  \\ 
                    & This alliance requires \$10 billion in annual U.S. defense spending.  \\ 
                    & This alliance requires \$15 billion in annual U.S. defense spending.  \\ 
                    
Conditions on Support  & The alliance treaty promises military support in any conflict. \\ 
                    & The alliance treaty promises military support only if this country did not provoke the conflict. \\ 
                    & The alliance treaty promises military support only if the conflict takes place in this country's region. \\
                    
Defense Cooperation & None. \\ 
                    & The alliance treaty provides basing rights for U.S. troops. \\
                    & The alliance treaty includes a shared military command. \\
                    & The alliance treaty includes an international organization to coordinate defense policies.  \\ 
% Issue linkages                    
Related Cooperation & None. \\
                    & The alliance is linked to greater trade and investment with the United States. \\ 
                    & The alliance is linked to greater support for the United States in the United Nations. \\ 
                    
Region              & Europe. \\ 
                    & Africa. \\
                    & The Middle East. \\ 
                    & Asia. \\   
                    & The Americas. \\ 
                                                                            
\hline \\
\end{tabular}
\end{adjustbox}
\caption{Table of alliance attributes in conjoint experiment profiles. I use the same set of attributes as treatments in the alliance formation and maintenance experiments.} 
\label{tab:conjoint-vars}
\end{table}


% summarize table
As \autoref{tab:conjoint-vars} shows, I include many salient alliance attributes.
Support or opposition from Republican and Democratic Senators, the Joint Chiefs of Staff, and the Secretary of State provide elite cues from elected officials, military leaders and diplomats. 
Other attributes cover key alliance characteristics such as trade ties, regime type, shared threat, military capability, conditions on support, defense cooperation, and issue linkages.
The regime type indicator includes nondemocracy, fragile democracy, and consolidated democracy. 
The financial cost values reflect the most conservative association between an alliance commitment and U.S. military spending from \citet{AlleyFuhrmann2021}. 
I also randomize the region of the hypothetical alliance partner to mitigate confounding.  


% Justify number of attributes
Each hypothetical alliance has fourteen attributes.
This ensures that attributes do not mask one another, but also that respondents are not overwhelmed and reduce the effort they put into assessing the full profile.
Studies of satisficing in conjoint experiments suggest that including fourteen attributes in a profile is unlikely to reduce data quality \citep{Bansaketal2019}. 
Furthermore, there is little evidence of satisficing when respondents are asked to rate or compare five profiles \citep{Bansaketal2018}, as is the case in this study.\footnote{I find little difference in treatment effects across rating tasks, which suggests that satisficing is not a major concern.} 


To analyze the results, I first estimate unconditional average marginal component effects.\footnote{In the appendix, I consider an alternative distributions of alliance profiles for this analysis, using the model-based population AMCE method of \citet{delaCuestaetal2021}.}
After that, I estimate interactions between the conjoint treatments and individual concerns.
The subgroup analysis tests whether partisanship and foreign policy dispositions modify the impact of elite cues and alliance characteristics. 
To analyze alliance support in different groups, I compare the marginal means of support across different groups and employing omnibus F-tests to assess aggregate differences \citep{Leeperetal2020}. 
Marginal means estimate average choices or ratings for each experimental attribute level, averaging over all other treatments. 



\subsection{Sample and Individual Measures}


There are two experiments--- one for alliance formation and another for maintenance. 
Each nationally representative sample contains 1500 U.S. respondents, recruited through Lucid Theorem.
These results will be under powered for very small effects, but have enough power to pick up large differences and interactions. 


For each respondent, I measured key individual correlates of alliance attitudes, especially partisan affiliation\footnote{I classified independent ``leaners'' as Democrats or Republicans, respectively. I coded pure independents or others that expressed no partisan lean as independents.} and foreign policy dispositions. 
I used standard batteries to measure internationalism and militant assertiveness \citep{Herrmannetal1999, KertzerBrutger2016}.
These pretreatment measures structure the subgroup analyses. 


Analyzing subgroups in the conjoint experiments requires categorical measures of foreign policy dispositions and partisanship. 
To divide respondents into isolationists and internationalists, I coded agreement with the most common survey measure as isolationism, and disagreement or a neutral stance as internationalism. 
The hawkishness index sums three questions about the use of force and war. 
Hawkish individuals scored above the midpoint of three on this scale, while doves scored three or lower. 
Finally, I interacted party affiliation, hawkishness and isolationism to analyze foreign policy dispositions within partisan groups.


\section{Results} 


In these results, I first present the unconditional average marginal component effect (AMCE) of elite cues and alliance characteristics.
I then consider how partisan identification, hawkishness and isolationism shape alliance attitudes. 
I find that elites have substantial power to lead alliance attitudes, but the consequences and extent of their leadership depends on foreign policy dispositions.
In addition to important differences in baseline alliance support, subsets of both parties hold rigid opinions. 
\autoref{fig:joint-plot} presents the AMCE of elite cues and alliance characteristics on individual choices in the alliance formation and maintenance experiments.
Given the large number of factors, all results figures highlight the most salient AMCE estimates.\footnote{See the choice and rating AMCE figures in the appendix for a full presentation of all the estimates in a single figure.}


The unconditional AMCE estimates show significant elite influence on alliance attitudes. 
Elite cues clearly increase public support for alliance formation and maintenance. 
Support from Senators and the Joint Chiefs of Staff is especially influential.
Backing from the Secretary of State increases support for alliance formation and has a smaller positive effect on alliance maintenance choices. 


\begin{figure}
	\centering
		\includegraphics[width=0.95\textwidth]{../figures/joint-amce-plots.png}
	\caption{Average marginal component effect of elite cues and alliance characteristics on public support for forming or maintaining a hypothetical military alliance. Feature names in parentheses. Components marked with abbreviated labels and some attributes omitted to make the plot more legible.}
	\label{fig:joint-plot}
\end{figure}


Some alliance characteristics influence alliance attitudes as well. 
Allied regime type is particularly consequential. 
Established democracy increases support for alliance formation and maintenance.  
Weak democracies are marginally more likely than non-democracies to receive public support, but these receive more limited public backing.
The magnitude of the established democracy AMCE is comparable to elite cues. 


Issue linkages also encouraged support for alliance formation and maintenance. 
Linkages to trade and investment with the United States increased support for alliance participation, relative to an alliance with no linkages. 
Political issue linkages in the United Nations bolstered individual support though this effect is smaller than that of trade. 
This suggests that issue linkages can facilitate new alliance agreements \citep{Poast2012} and bolster alliance credibility \citep{Poast2013} by boosting popular support. 


Last, alliance context and costs shift alliance attitudes. 
Trade ties and serious common threat encouraged support for alliance maintenance. 
Relative to the lowest annual military spending cost, annual costs of \$10 billion or more decreased support for upholding an alliance.  
Respondents also viewed alliances in Europe or the Americas more favorably than commitments to African states. 


The above results assume that individuals respond in the same way to different cues and alliance characteristics. 
But individual concerns, especially the confluence of partisanship and foreign dispositions, structure alliance attitudes.
Examining these factors shows who elites lead.  



\subsection{Partisanship, Hawkishness, Isolationism, and Alliance Attitudes}


Leading models predict that individuals will respond primarily to co-partisan or other trusted elites. 
The partisanship that gives elites some of their influence is bound up with foreign policy dispositions that also shape alliance attitudes, however. 
In this section, I estimate support for alliances across respondents with different partisan affiliations and foreign policy dispositions.  


This analysis helps establish who leads in public opinion towards alliances. 
If elite cues exert little impact or only push respondents in ways that match individual predispositions, they are more likely to follow public opinion. 
On the other hand, if elites cues increase alliance support among likely alliance skeptics and decrease support among likely backers, elites lead. 


In the following, I plot the marginal means of support for alliance participation for distinct foreign policy dispositions within the two major parties under key conjoint treatments.  
I start with the marginal means of support given different elite cues. 
I then examine how respondents with different partisan affiliations and foreign policy dispositions view key alliance characteristics. 


\subsubsection{Elite Cues}


\autoref{fig:party-dispo-form-el} and \autoref{fig:party-dispo-main-el} show the marginal means of support for alliance formation and maintenance across partisan and foreign policy disposition subgroups.\footnote{See the appendix for details on the distribution of foreign policy dispositions across party identification.} 
Each panel plots the marginal mean of support for every categorical combination of militant assertiveness and internationalism within both parties. 
Both figures capture how individuals in each group respond to elite cues. 


Alliance attitudes and responses to elite cues reflect a complex combination of foreign policy dispositions and partisanship. 
The same foreign policy dispositions have distinct implications for alliance attitudes among Republicans and Democrats, so partisanship matters.
At the same time, foreign policy dispositions produce substantial differences in alliance attitudes within parties.  


\begin{figure}[htpb]
	\centering
		\includegraphics[width=0.95\textwidth]{../figures/party-dispo-form-el.png}
	\caption{Marginal means of support for forming a hypothetical alliances across party identification and foreign policy dispositions under different elite cues. For each group, the estimates mark the marginal mean of support for alliance participation under different alliance treatments. The vertical line highlights a marginal mean of .5, as it is a threshold for majority support. Components marked with abbreviated labels to make the plot more legible. Independents omitted.}
	\label{fig:party-dispo-form-el}
\end{figure}


Among Democrats and Republicans, hawkishness increases support for alliance participation. 
There are partisan differences in this relationship, however, as hawkish Democrats express higher support for alliance participation than hawkish Republicans. 
Hawkish and isolationist Democrats are the strongest supporters of alliance participation. 
Hawkishness also increases support for alliance participation among isolationists. 
Willingness to use force thus offsets general skepticism of international engagement in alliance attitudes. 


Isolationism alone does not reduce support for alliance participation, as one might expect.
Rather, alliance opposition comes from skeptics of international engagement and the use of force. 
Isolationist and dovish individuals are the greatest skeptics of alliance formation and maintenance. 
The combination of isolationism and limited support for military force is at the heart of alliance opposition. 
Although Republican doves are rare, they are integral to alliance skepticism in the GOP, especially because they disregard elite cues. 
Dovish Democrats are also more likely to oppose alliances.  


\begin{figure}
	\centering
		\includegraphics[width=0.95\textwidth]{../figures/party-dispo-main-el.png}
	\caption{Marginal means of support for maintaining a hypothetical alliances across party identification and foreign policy dispositions under different elite cues. For each group, the estimates mark the marginal mean of support for alliance participation under different alliance treatments. Vertical line highlights a marginal mean of .5, as it is a threshold for majority support. Components marked with abbreviated labels to make the plot more legible. Independents omitted.}
	\label{fig:party-dispo-main-el}
\end{figure}


In addition to shifting baseline alliance attitudes, foreign policy dispositions change individual responses to elite cues. 
Isolationists are less likely to heed elite cues. 
Attitudes towards international engagement are especially important for elite influence on Democrats. 
Internationalist Democrats respond strongly to support from Democratic Senators, and also look to cues from the Secretary of State and Joint Chiefs of Staff. 
Hawkish and isolationist Democrats express high and rigid support for forming and maintaining alliances. 


Among Republicans, hawks are most receptive to elite cues. 
Regardless of their view of international engagement, there are clear differences in the marginal means of alliance support for hawkish Republicans based on Republican Senate support or opposition.
Hawkish Republicans also follow cues from military elites. 
As a result, Republican elites can lead alliance attitudes among individuals who are disposed to support forceful international engagement and thereby constrain alliance support among the most likely alliance backers in their party. 
The gap in hawkish Republican attitudes from differences in Republican elite support is especially pronounced in the alliance formation experiment. 
In the reverse of the Democratic party, the most likely alliance supporters in the Republican party hold more plastic alliance attitudes. 



\subsubsection{Alliance Characteristics}



% partisan differences in democracy
In addition to elite cues, foreign policy dispositions and partisanship mold individual responses to alliance characteristics. 
\autoref{fig:party-dispo-form-char} and \autoref{fig:party-dispo-main-char} summarize partisan and dispositional differences in alliance characteristics treatments from the formation and maintenance experiments. 
Both figures present the marginal mean of alliance support for Republicans and Democrats with different foreign policy dispositions given some of the alliance characteristics treatments. 


Allied democracy is a point of bipartisan agreement among individuals with different foreign policy dispositions.
When allied democracy changes in the alliance formation experiment, Republicans and Democrats with identical foreign policy dispositions shift their attitudes in similar ways. 
Internationalists in both major parties value allied democracy. 


Democratic support for democracy in an ally is pronounced in the alliance maintenance experiment. 
Dovish Democrats are especially skeptical of alliances with non-democracies, relative to other treaties. 
Republicans express more mixed views of allied political regimes. 
While internationalist Republicans hold similar views to Democrats, isolationist Republicans place less weight on allied democracy, especially in alliance maintenance. 


\begin{figure}
	\centering
		\includegraphics[width=0.95\textwidth]{../figures/party-dispo-form-char.png}
	\caption{Marginal means of support for forming a hypothetical alliances across party identification and foreign policy dispositions under different alliance characteristics. For each group, the estimates mark the marginal mean of support for alliance participation under different alliance treatments. Vertical line highlights a marginal mean of .5, as it is a threshold for majority support. Components marked with abbreviated labels and some attributes omitted to make the plot more legible. Independents omitted.}
	\label{fig:party-dispo-form-char}
\end{figure}



\begin{figure}
	\centering
		\includegraphics[width=0.95\textwidth]{../figures/party-dispo-main-char.png}
	\caption{Marginal means of support for maintaining a hypothetical alliances across party identification and foreign policy dispositions under different alliance characteristics. For each group, the estimates mark the marginal mean of support for alliance participation under different alliance treatments. Vertical line highlights a marginal mean of .5, as it is a threshold for majority support. Components marked with abbreviated labels and some attributes omitted to make the plot more legible. Independents omitted.}
	\label{fig:party-dispo-main-char}
\end{figure}


% Some points of consistency
Support for international engagement also produces some partisan overlap in alliance attitudes. 
Internationalist individuals are more responsive to trade and foreign policy issue linkages than isolationists. 
In the alliance maintenance experiment, dovish and internationalist individuals are more likely to reduce their support as alliance costs increase.


% other differences in region
Both experiments uncover partisan differences in attitudes towards alliances in different regions. 
Hawkish Republicans oppose alliances with African countries and prefer alliances in Europe or the Americans. 
Democrats with similar foreign policy dispositions express no regional preferences.  


% highlight differences in formation and maintenace
All of the above results reveal that forming new alliances has lower baseline support than maintaining existing treaties, so elite cues are crucial for new alliances. 
Alliance formation often commands narrow majority support on average.
Dovish isolationists are particularly opposed to new alliances, though elites can persuade Democrats with this disposition. 
Whether elites support or oppose an alliance determines whether it has majority or minority support within each party. 


Alliance maintenance commands more robust support. 
Regardless of alliance characteristics or elite cues, the marginal means of support for alliance maintenance are almost all above .5. 
Even dovish isolationists offer split verdicts on alliance maintenance.


% wrap up
These results suggest that elite cues, partisanship and foreign policy dispositions interact to shape alliance attitudes.
Partisanship and foreign policy dispositions set the starting point from which elite cues guide most Americans. 
There is an important partisan asymmetry in alliance attitudes as well. 
Democrat leaders can lead alliance skeptics and have less influence over the most committed alliance supporters. 
Republican elites lead alliance supporters, but cannot persuade committed alliance skeptics. 
As a result, elite cues exert substantial influence on the majority of both parties, but their impact depends on foreign policy dispositions. 


An online appendix provides further details supporting these results. 
In the appendix, I examine marginal means by partisanship and foreign policy dispositions, consider alternative assumptions about alliance profile distributions for the unconditional AMCE estimates, analyze responses to an open-ended question, and compare results with the continuous rating measure of alliances to inferences from the choice question.
All these checks are consistent with these findings. 


\section{Discussion and Conclusion} 


% Overview
I find that elites often lead public alliance attitudes, though some individuals hold rigid attitudes. 
Partisans respond largely to cues from co-partisan elites, but the magnitude of their response depends on hawkishness and isolationism. 
Alliance democracy, issue linkages, shared threat, financial cost and trade have noticeable impacts as well.  


% thus, net on puzzle
Elites have substantial latitude to lead public opinion on alliances, but foreign policy dispositions constrain their influence. 
The most committed alliance supporters ---hawkish isolationist Democrats--- pay little attention to elite cues.
Similarly, elite cues have no impact on the most committed alliance skeptics --- dovish and isolationist Republicans. 
The result is a partisan asymmetry in elite leadership of alliance attitudes. 
Republicans can lead copartisan alliance supporters, while Democrats can lead copartisan alliance skeptics. 


% bring it in on observed alliances
These findings have four key implications for understanding public attitudes towards alliances like NATO. 
First, the Republican and Democratic parties contain committed groups of alliance skeptics and supporters, respectively.
Outside these groups and independents, most Americans follow elite cues in forming alliance attitudes. 
This makes how much elites follow fixed alliance attitudes in their party a critical issue, because following fixed attitudes will move the two parties in competing directions. 
At the same time, elite opposition rarely pushes alliance attitudes into majority opposition to existing treaties. 


Second, allied democracy will bolster continued public support for alliances.
Individuals in both parties prefer alliances with democracies. 
Democratic backsliding in U.S. allies could therefore undermine public support for alliances, especially by giving skeptical elites a powerful angle for criticism. 


Third, my findings support the view that cycles in public opinion could make democratic commitments less reliable \citep{GartzkeGleditsch2004}. 
Although the results suggest that many members of the public hold considered opinions \citep{PageShapiro1992}, they also show that elites have substantial influence. 
One set of elite cues alone can push aggregate support from a solid majority to a evenly divided issue in their party.
Public opposition from military or diplomatic elites would also bolster the impact of skeptical politicians and cut public support. 


% why NATO robust under Trump? 
Finally, these results help us understand public opinion towards alliances like NATO during the Trump administration.
Although Trump often criticized U.S. allies, alliance commitments usually commanded majority support throughout his administration \citep{PewNATO2020}. 
Some of this reflects Democrats' aversion to Trump, but it also reflects high baseline support for alliances, even in the Republican party.  
For many Republicans, hawkishness offsets isolationism in alliance attitudes.
Although Trump likely increased Republican skepticism of alliances, concern that he would inspire resurgent isolationism in the Republican party may have been overstated.
Isolationist and dovish Republicans did not change their alliance attitudes in response to Trump's rhetoric, as his rhetoric matched their views. 


% limitations
These findings have some limitations. 
For one, while the sheer variety of alliances means that the above profiles are plausible, generalizing from survey experiments is challenging. 
The artificial nature of a survey experiment provides the necessary control to disentangle different sources of public attitudes, but no hypothetical alliance can fully reflect real world commitments.


% can't show pandering
While this paper provides new insight into the question of who leads and who follows, it does not give a comprehensive account of the issue. 
How much and when elites decide to follow fixed alliance attitudes in their party falls outside the scope of this paper. 
Having identified who elites can lead, understanding the long-run dynamics of leading and following is a crucial subject for future research. 


% non-us results- might be different, subject for future inquiry
This study also focuses on the United States, which has an unusual alliance network. 
Though public opinion towards alliances in the United States is important, attitudes in other countries matter as well. 
Future research should examine the sources of alliance attitudes in other countries. 


% future stuff: feedback, content of elite cues
These results provide a foundation for further inquiry into the domestic politics of military alliances. 
Two questions are especially interesting in this respect.
First, how much feedback takes place between public opinion and elite cues? 
When do politicians follow rigid alliance attitudes or lead the rest of their party in a competing direction? 
Politicians might view marginal changes in opinion due to changes in threat or allied democracy as an opportunity to encourage or arrest further changes in public support.
Second, would leaders face significant public disapproval if they withdrew from an alliance? 
This study focused on generic support, but future research could build on \citet{TomzWeeks2021} and examine specific alliance policy changes. 


These questions address how elites form and maintain domestic coalitions around international engagement. 
In the 75 years since the end of World War II, shifting elite cues, partisanship, generational experiences and allied characteristics may mean that different groups back alliances today than in 1950. 
Tracking changes in the domestic coalitions backing alliances is another worthwhile task for future research.


% wrap it up 
In conclusion, public opinion towards alliances is largely a function of elite cues, but elites do not lead the whole electorate.  
Important subsets of both parties hold rigid alliance attitudes. 
Alliances attitudes are therefore a complex mixture of elite cues and individual considerations. 



\newpage

% Bibliography
 
\bibliography{../../MasterBibliography} 




\end{document}
