\documentclass[12pt]{article}

\usepackage{fullpage}
\usepackage{graphicx, rotating, booktabs} 
\usepackage{times} 
\usepackage{natbib} 
\usepackage{indentfirst} 
\usepackage{setspace}
\usepackage{grffile} 
\usepackage{hyperref}
\usepackage{adjustbox}
\usepackage{amsmath}
\usepackage{siunitx}
\usepackage{multirow}
\setcitestyle{aysep{}}


\singlespace
\title{\textbf{Elite Cues and Public Attitudes Towards Military Alliances}}
\author{Joshua Alley \\
Postdoctoral Research Associate \\
University of Virginia.\thanks{Thanks to Erik Lin-Greenberg, Philip Potter, Justin Schon and Todd Sechser, as well as participants in the Democratic Statecraft Lab Research incubator, the Lansing B. Lee/Bankard Seminar in Global Politics, 2020 Annual Meeting of the Peace Science Society and 2021 Meeting of the International Studies Association for helpful comments. This project was reviewed by the University of Virginia IRB (Protocol 3866) and preregistration files for this study are hosted in an OSF repository at https://osf.io/g28zs.} \\
jkalley@virginia.edu
}
\date{}

\bibliographystyle{apsr}

\begin{document}

\maketitle 

\doublespace 

\begin{abstract}
Do elite cues exert extensive, limited or minimal influence on public support for military alliances in the United States? 
In this article, I demarcate the boundaries of elite influence on public opinion towards alliances by examining whether partisanship and foreign policy dispositions modify individual responses to elite cues.
Partisanship, isolationism and militant assertiveness could determine which elites voters trust and establish baseline alliance dispositions. 
I then use two conjoint survey experiments to examine the roots of public attitudes towards forming and maintaining international alliances.  
I find that elites can lead most of the electorate, but subsets of both major parties hold rigid attitudes. 
These fixed attitudes have a partisan asymmetry, as staunch alliance supporters in the Democratic party and consistent alliance skeptics in the Republican party both ignore elite cues.  
Elites can lead most public opinion towards military alliances, but individual concerns sometimes constrain their influence.  
\end{abstract}


\newpage 


\section{Introduction}


% lay out the question
Do elite cues impact U.S. public opinion towards military alliances, and if so, who do elites lead?
Although elite influence on public foreign policy attitudes is well-established, the relationship between elite cues and public alliance attitudes is unclear.
For example, many observers feared that Donald Trump's rhetoric would undermine domestic support for alliances, yet U.S. public approval of alliances like NATO changed little during the Trump administration \citep{PewNATO2020}. 
This difficulty occurs because like many foreign policy issues, alliance attitudes are subject to the longstanding puzzle of who leads whom in public opinion on foreign policy.
Elite support or opposition to alliances could lead or reflect public opinion.\footnote{This article considers the leading or following question for Trump and NATO: \url{https://fivethirtyeight.com/features/is-trump-fueling-republicans-concerns-about-nato-or-echoing-them/}}


Elite cues could exercise extensive, limited or minimal influence on public alliance attitudes. 
First elite rhetoric could lead public opinion on alliances, given limited public information and interest in foreign policy \citep{Canes-Wrone2006, BaumPotter2008, Druckman2014}.
Alliance politics have low public salience, which likely increases elite influence. 
On the other, leaders often conform their rhetoric to public attitudes, so elites might have minimal influence \citep{Barberaetal2019, HagerHilbig2020}.
Even when the public pays little attention to international affairs, their opinions have consistency and structure \citep{Holsti1992, PageShapiro1992}.
Individual foreign policy dispositions like isolationism and militant assertiveness \citep{KertzerZeitzoff2017} could establish alliance attitudes for elite cues to match.
Last, perhaps elites have limited influence, as some of the public follows their cues, while others do not. 


The extent of elite leadership shapes the relevance of public opinion in alliance politics.
If elites lead public opinion, then public attitudes are less likely to constrain elite alliance formation and maintenance decisions.
But if public opinion is unresponsive to elite cues, then public attitudes might constrain efforts to form new alliances or withdraw from existing treaties. 
%Thus, the impact of elite cues is crucial to understanding U.S. alliance formation and maintenance.  
Despite the importance of this issue, we do not know why the public supports or opposes alliance commitments. 
Most evidence comes from opinion polls measuring public sentiment towards alliances like the North Atlantic Treaty Organization (NATO).
These polls provide useful data, but they cannot connect elite cues and public attitudes.


% contribution
I examine the extent of elite influence on U.S. alliance attitudes by assessing whether foreign policy dispositions and partisanship change individual responses to elite cues.
How co-partisan elite cues impact individuals with different predispositions towards alliances from isolationism and militant assertiveness establishes the bounds of elite leadership, as partisanship and foreign policy dispositions set the starting point from which elite cues alter alliance attitudes. 
Isolationists are skeptical of alliances, while hawks often back alliance participation. 
If co-partisan elite cues encourage isolationists to support alliances and hawks to oppose alliances, and thus sway public opinion regardless of initial dispositions, elites exert extensive influence. 
But if co-partisan elite cues have no impact on all or substantial parts of the electorate, their impact is minimal or limited.


% Assess w/ a survey experiment
To provide causal evidence on elite leadership of public opinion towards alliances, I use two conjoint survey experiments.
This allows me to randomize many alliance characteristics and elite cues \citep{Hainmuelleretal2014}.
Unlike in observational data, in an experiment that randomly assigns elite cues, I can leverage information on foreign policy dispositions within parties to distinguish who elites can lead. 
The first experiment scrutinizes attitudes towards alliance formation, while the second addresses alliance maintenance. 


% findings: 
In two nationally representative samples, I find evidence of extensive elite leadership, with slight but important limits.
While most individuals follow co-partisan elite cues regardless of their foreign policy dispositions, the strongest Democrat alliance supporters have rigid alliance attitudes, as do staunch Republican alliance skeptics. 
Partisanship and foreign policy dispositions also set the base from which elite cues move alliance attitudes.
Elite cues thus exert broad influence, but their impact depends on partisanship, hawkishness and isolationism.
Topline alliance support numbers thus reflect substantial variation in alliance attitudes within the U.S. electorate and both major parties.


I find a partisan divide in rigid alliance attitudes.
Hawkish and isolationist Democrats are robust alliance supporters.
Dovish and isolationist Republicans are committed alliance skeptics. 
Therefore, Republicans can lead the most likely alliance supporters in their party, while Democrats can lead relative alliance skeptics. 
If elected leaders follow the strongest alliance attitudes in their party, they will polarize the rest of the electorate. 


%% partisan differences in democracy, region
%A few alliance characteristics also impact public attitudes. 
%Democrats and most Republicans prefer alliances with other democracies to supporting nondemocracies. 
%Issue linkages increase public support for alliance participation, while high financial costs reduce support for alliance maintenance. 
%Some Republicans express strong regional preferences, with minimal interest in alliances with African and Middle Eastern states. 


% differences between formation and mainteance
There is also a gap between public support for alliance formation and maintenance.
Even with elite opposition, upholding existing alliances almost always retains majority support. 
In alliance formation, elite cues determine whether a new treaty has majority or minority support. 
Therefore, elites have more influence over making new alliances than changing existing commitments. 


% Importance part 1: public opinion undergirds alliance com in democ
In addition to providing new insight into debates over elite leadership of public opinion, there are three reasons that understanding U.S. public opinion towards alliances is worthwhile. 
To start, public opinion is central to debates over whether democracies make more reliable commitments than other states.\footnote{Public opinion is important, but it is not deterministic. \citet{Kreps2010} notes that public disapproval may not hinder coalition warfare, especially when elite consensus favors fighting.} 
If public opinion towards alliances is indifferent to elite cues, stable attitudes and reliable commitments follow \citep{Gaubatz1996}.
If elite cues drive public opinion, then public attitudes can shift quickly, leading to cycles that hinder democratic reliability \citep{GartzkeGleditsch2004}.
NATO leaders often worried that changing public attitudes would undermine the alliance \citep{Sayle2019}.  


% importance part 2: practical relevance- US role in world. 
The impact of elite cues on alliance attitudes also speaks to the consequences of a prominent scholarly and policy debate. 
Two competing visions of U.S. foreign policy depend on alliances. 
One perspective believes that the United States should reduce its alliance commitments to pursue a restrained grand strategy \citep{Preble2009, Posen2014}.
The other argues that continued deep engagement through alliances is the best way to promote U.S. security and prosperity \citep{Brooksetal2013, BrandsFeaver2017}. 
If elite cues have extensive influence, leaders will face fewer public constraints on implementing their grand strategic vision. 


% importance part 3: WHY support international cooperation, not just consequences of international institutions
In addition to its practical importance, this study fills a gap in international institutions scholarship. 
Scholars are more likely to study how international institutions affect public attitudes (e.g. \citep{KayaWalker2014, Greenhill2020}), than scrutinize the sources of public attitudes towards international institutions themselves. 
Other studies use observational survey data to examine public opinion towards international cooperation in multilateral financial institutions \citep{Edwards2009} or the United Nations \citep{Torgler2008, DellmuthTallberg2015}. 
That leaves limited causal evidence on why individuals hold particular alliance attitudes.
In one study of public opinion and military alliances, \citet{TomzWeeks2021} address a different question by showing that the presence of an alliance increases public support for foreign military intervention. 
\citet{Chuetal2021} explore how values and interest based elite cues shape public attitudes towards alliance maintenance. 
I build on these works with more general experiments on alliance formation and maintenance that clarify the reach of elite cues and account for many alliance characteristics. 


% Implications
The finding that elites have substantial influence on public attitudes towards alliances, but some individuals hold rigid opinions has important implications for the future of U.S. alliance politics. 
Although elite cues affect public support for U.S. alliances, they do not reach the whole electorate.
Also, one set of elites alone cannot produce majority opposition to existing treaties, because alliance maintenance commands substantial support.
Bipartisan opposition to alliances could reduce public support enough for leaders to withdraw from an alliance with little backlash, however.  
Unlike existing treaties, public support for new alliance commitments is very responsive to elite cues. 
Therefore, whether political elites follow the contradictory fixed attitudes in the two major parties will shape domestic support for U.S. international engagement through alliances.

% Try it w/o the plan of paper for flow. 


\section{Elite Leadership and Alliance Attitudes}


Public opinion molds democratic foreign policy and alliance politics in several ways.
First, it affects military intervention decisions \citep{Tomzetal2020, LinGreenberg2021}. 
In democracies, anticipation of paying public audience costs for treaty violation encourages limited promises of military support \citep{Chibaetal2015, FjelstulReiter2019}. 
Moreover, public attitudes are central to disputes about the reliability of democratic commitments \citep{Gaubatz1996, GartzkeGleditsch2004}. 
Policymakers also track public support for alliances \citep{Sayle2019}. 


% NATO example to bring out the puzzle
In addition to its importance, there is meaningful variation in pubic opinion towards military alliances. 
\autoref{fig:nato-op-time} plots the percentage of respondents supporting NATO in 59 surveys from 1974 to 2020.\footnote{These surveys ask respondents to assess NATO in many ways. I consider favorable opinions, feeling thermometer ratings of 50 or higher, and support for increasing or maintaining U.S. commitment as indicators of support for NATO.} 
Most surveys show majority support for NATO, but average support fell after 2000.  


\begin{figure}
	\centering
		\includegraphics[width=0.95\textwidth]{../figures/nato-op-time.png}
	\caption{US public support for NATO from 1974 to 2020. Each point marks a unique poll, and colors differentiate the percentages of respondents that expressed support, opposition or neutral/no opinion of NATO. Loess lines estimate the average support for each group in every year. Topline data from the Roper Center's iPoll database.}
	\label{fig:nato-op-time}
\end{figure}


Observed alliance attitudes like those in \autoref{fig:nato-op-time} are subject to a longstanding puzzle in public opinion on foreign policy--- who leads whom? 
Does public support for NATO follow elite cues, or do elite cues reflect public attitudes?  
Put differently, if we observe elite and public support for alliances, it is unclear if public attitudes follow elite cues or if established public attitudes drive elite cues. 
Both perspectives offer plausible models. 


Evidence on whether elites lead or follow public opinion is divided.
Some suggest that elites are more likely to lead public opinion. 
\citet{Canes-Wrone2006} finds that U.S. Presidents rarely follow public preferences if they disagree, and have ample freedom to lead foreign policy attitudes. 
\citet{JacobsShapiro2000} argue that elites track public opinion to manipulate it, not conform to it. 
\citet{Kreps2010} notes that public disapproval did not constrain participation in NATO's International Security Assistance Force in Afghanistan. 
Moreover, foreign policy is a secondary concern for many voters, so elite foreign policy views and rhetoric can diverge from public attitudes with few political repercussions \citep{BusbyMonten2012}. 


Other findings suggest that elites conform their rhetoric and policy stances to public opinion. 
\citet{Barberaetal2019} use social media data to show that legislators are more likely to follow than lead public opinion, including on some foreign policy issues. 
\citet{HagerHilbig2020} find that exposure to public opinion research moves speech and policy positions by German politicians closer to majority opinion. 
\citet{GuisingerSaunders2017} claim that for issues with low partisan polarization, information effects dominate public opinion, though elite cues matter more for polarized issues like cap and trade schemes. 
\citet{Haesebrouck2019} uncovers little evidence that European elites led public support for military interventions in Libya and the Islamic State. 
\citet{Bechteletal2015} find that elite cues and frames led Swiss individuals, especially those with low knowledge, to reinforce their prior immigration attitudes. 
Even military elites with no electoral concerns shape their recommendations in response to public opinion \citep{LinGreenberg2021}. 



Alliance attitudes are subject to this puzzle or who leads whom. 
Elites could exert extensive, minimal or limited influence on alliance attitudes. 
On the one hand, limited public information about alliances could lead to broad elite influence \citep{Druckman2001}. 
On the other, the public opinion towards alliances may depend on individual concerns, including foreign policy dispositions, as these intuitions about international affairs provide consistent heuristics even with limited information \citep{Herrmannetal2009, KertzerZeitzoff2017}.
A combination of the two is possible, as some alliance attitudes may be more plastic than others, leading to limited influence.  
\citet{PageShapiro1992} note that public opinion is broadly consistent and rational, and changes in predictable ways in response to information from multiple sources, including elite cues. 


Understanding alliance attitudes therefore speaks to a fundamental debate about public opinion on foreign policy.  
In the following, I attempt to establish how elite cues affect public opinion towards alliances.\footnote{This does not fully address leading or following, as I do not show what drives elite cues. Rather, I assess a crucial component of elite leadership that cannot be inferred from observational data.}
In doing so, I attempt to demarcate the boundaries of elite influence by assessing whether all, some, or little of the electorate responds to elite cues.
The remainder of this argument explains how partisanship and foreign policy dispositions provide leverage to understand who holds rigid or plastic alliance attitudes, and which responses to elite cues reflect extensive elite leadership. 
To begin, I outline how elite cues might lead public opinion in general. 


\subsection{Elite Cues} 
% Framing/elite leading

Elite cues are a plausible determinant of alliance attitudes. 
Under this general model the public follows trusted elites in forming their opinion, so elite portrayals of alliances bolster or undermine public support.
Thus, public opinion towards alliances permeates down from the top and is endogenous to elite views \citep{Druckman2014}.
There is substantial evidence that elites influence public foreign policy attitudes \citep{BaumPotter2008}. 
The media often convey elite cues and frames.
Social media may further amplify elite influence \citep{BaumPotter2019}.   


Information shortcomings make individuals more responsive to elite framing and cues \citep{Druckman2001, Peterson2017} and the public has limited foreign policy information \citep{BaumPotter2008}.
Furthermore, alliance politics has low salience within foreign policy. 
Alliances are less prominent than international conflict, which is the most common subject in studies of foreign policy opinions. 
Therefore, elite support or opposition could shape alliance attitudes because individuals rely on trusted elites in an issue environment with little alternative information. 


% cue-giver matters 
Multiple elites can give public alliance cues.
Elected officials, diplomats and military leaders all participate in alliance politics.
The public visibility and influence of elected leaders is well-established.  
Cues from military leaders can shape public opinion about the use of force \citep{Golbyetal2018}, so military endorsements may also move alliance attitudes. 
Diplomatic elites are high profile domain experts. 
Public perceptions that military leaders and diplomats are well-informed about alliances will likely increase their influence. 


% highlight partisanship
In an elite cues model, support for alliances by trusted elites should increase individual support for alliances, and elite opposition will reduce support.   
Partisanship further shapes elite influence by establishing trust.
Under partisan polarization, individuals distrust and discount messages from out-partisan elites. 
Conversely, trust makes cues from co-partisan elites more influential \citep{Druckmanetal2013}.
As a result, unified elite cues will have a large impact.
For example, \citet{Berinsky2007} finds that unified elite support for war leads to robust public support. 


% transition paragraph: scope of influence
Elite cues offer a straightforward and compelling explanation of alliance attitudes.
Even information about alliance characteristics like allied democracy or military spending likely reaches the public through elite sources. 
Extensive elite influence on alliance attitudes is plausible. 
When they receive elite messages, individuals also hold prior attachments, intuitions and beliefs, however.
Individual foreign policy dispositions and partisanship set initial alliance dispositions and could modify how rigid or plastic alliance attitudes are under elite cues. 


\subsection{Foreign Policy Dispositions and Partisanship}


% Overview para
Foreign policy dispositions and partisanship shape individual perceptions of international cooperation. 
These individual concerns have two consequences for alliance attitudes. 
First, they establish individuals' baseline alliance support, or willingness to back alliances in general.\footnote{Another way to think of baseline support is an individual disposition to support an average or typical alliance.} 
Second, individual concerns might change individual responses to elite cues. 
Individuals might hold prior attachments so tightly that elite cues have minimal influence, or alliance predispositions could make some individuals responsive and others unresponsive, resulting in limited elite influence.



% foreign policy disposition
Foreign policy dispositions are intuitions about international politics. 
These principles shape how people respond to decisions such as backing down from military intervention threats \citep{KertzerBrutger2016}. 
Militant assertiveness and internationalism are two key foreign policy dispositions \citep{Herrmannetal1999} that could impact alliance attitudes. 


% internationalists more likely
% Define internationalism  
Internationalism is an inclination to engage with other countries and contribute to international endeavors. 
Internationalists support U.S. involvement in foreign affairs.
As such, they are more likely to favor alliance commitments. 
Conversely, isolationists are skeptical of international institutions and cooperation, dislike foreign involvement and prioritize domestic affairs \citep{Kertzer2013}. 
As a result, isolationists should dislike alliances.  


% ample evidence of isolationism 
Past isolationist opposition to alliances is well documented. 
Isolationist senators like Robert Taft were the core of U.S. opposition to ratifying NATO \citep{Kaplan2007}.
The U.S. tradition of distrust towards ``entangling alliances'' that only broke after World War II \citep{Kupchan2020}.


% militant assertiveness 
Militant assertiveness reflects individual views of using force to address international problems \citep{Herrmannetal1999}. 
Dovish individuals are low on militant assertiveness and prefer nonviolent policies.
Hawkish individuals are more willing to employ force.
Although alliances are cooperative institutions that attempt to deter conflict, they also aggregate military capability and obligate members to fight.
Military intervention obligations should make alliances less appealing to doves.
A general skepticism of using military force in general will make doves less likely to support military alliances.  
European pacifists are some of the most consistent opponents of NATO, for example \citep{Thies2015}.


Unlike doves, I expect that hawks value capability aggregation through alliances and are more willing to hazard foreign wars and capability aggregation. 
Committing to fight and investing defense is less of an issue for hawkish individuals. 
In-group loyalty is a key source of militant assertiveness \citep{Kertzeretal2014} and could increase support for alliance maintenance and participation by emphasizing group cohesion in the face of external pressures.


% dispositions set prior attitudes
Internationalism and militant assertiveness set individual inclinations towards alliances before receiving elite cues.\footnote{To give an example from a different domain, \citet{KertzerBrutger2016} leverage foreign policy dispositions to decompose audience costs into belligerence and consistency costs.}
Essentially, foreign policy dispositions create a baseline from which elite cues might impact alliance attitudes. 
Whether and how individuals respond to elite cues given their prior dispositions will provide insight into elite influence. 


% introduce/transition partisanship
In addition to internationalism and militant assertiveness, partisanship has an important role in alliance attitudes. 
First, party identification connects elite cues and individual concerns by determining whose cues matter.
Individuals look to cues from trusted elites, and partisanship is a straightforward heuristic for who to trust. 
Moreover, partisanship is correlated with foreign policy dispositions. 
Militant assertiveness and internationalism are correlated with partisanship. 
Conservatives in the United States have a longstanding history of isolationism \citep{Kupchan2020}.
Republicans are more hawkish than Democrats as well \citep{Gries2014}. 


% limits 
Understanding alliance attitudes thus requires careful attention to elite cues, partisanship and foreign policy dispositions. 
Although elites are likely influential, the extent of elite influence is unclear because partisanship changes individual perceptions of elite cues and is correlated with foreign policy dispositions that could shape individual dispositions towards alliances. 
Perhaps Republican leaders' opposition to alliances does not decrease Republican support for alliances, it reflects isolationism in the Republican party, for instance. 
Hawkish or isolationist individuals might also discount elite cues and rely on their disposition towards alliances. 



\subsection{The Boundaries of Elite Influencea}


% how distinguished
To assess elite leadership, I examine how elite cues impact Democrats and Republicans with different predispositions towards alliances.
Partisanship, militant assertiveness and isolationism create distinct individual inclinations to back or oppose alliance participation. 
These inclinations set baseline alliance attitudes. 
How elite cues move attitudes relative to baseline opinions then shows who elites lead. 
If co-partisan elite cues move most of the electorate regardless of foreign policy dispositions, elites exert extensive influence on alliance attitudes. 
If co-partisan elite cues move some attitudes, elite influence is more mixed. 
Minimal elite influence implies few individuals respond to elite cues. 


% no a priori about different combinations
I expect that internationalism and hawkishness increase baseline alliance support and individuals may be isolationist and hawkish, internationalist and hawkish, isolationist and dovish, or internationalist and dovish.\footnote{While some existing research does not divide isolationists into hawks and doves and distinguishes between cooperative and militant internationalists \citep{Kertzeretal2014}, I divide isolationists by hawkishness to assess the net impact of competing dispositions. To streamline discussion across the four categories, I do not use the terms cooperative and militant internationalism in the manuscript, though the concepts are present.}  
Because the different dispositions overlap their relative weight is an important concern.
Dovish isolationists are the most likely alliance skeptics, while hawkish internationalists are the most likely alliance supporters. 
I do not have strong priors about the relative strength of hawkishness and isolationism, however.
One effect could dominate the other, the two factors could offset, or they could interact in unexpected ways.\footnote{As a result, some of the following analysis is exploratory.}


% dividing partisans
Dividing respondents by partisanship and foreign policy disposition provides leverage over who holds plastic or rigid alliance attitudes. 
Individual concerns set the starting point from which elite cues and other information about an alliance might shift public attitudes.
This in turn allows me to identify the boundaries of elite leadership. 


% explain: isolation
Under extensive elite leadership, elite cues should change opinion regardless of individual predispositions to support or oppose alliance participation. 
For example, elite support will increase support for alliance participation even among isolationists. 
Similarly, if elite opposition reduces support among hawkish individuals who would otherwise support an alliance, elite cues lead public opinion. 
Such responses imply broad and direct elite influence. 


% explain: hawks 
If elites cues have no effect on alliance attitudes or only impact some of the population, then their leadership is more constrained.
Strong predispositions from isolationism and militant assertiveness could minimize the direct impact of elite cues, or limit it to part of the electorate.
Elites might still exercise indirect leadership by shaping alliance salience and presenting specific information, but null effects imply limited direct influence in the vein of classic elite cues arguments. 
\autoref{tab:arg-sum} summarizes what evidence is consistent with extensive elite leadership.  
If public attitudes follow elite cues, even when cues conflict with their likely disposition towards alliances, elites exert extensive leadership. 


\begin{table}[hbt!]
\begin{center}
\begin{tabular}{ccc}
Co-Partisan Elite Cue & \multicolumn{2}{c}{Foreign Policy Disposition}  \\
\hline 
           &          Internationalist & Isolationist  \\
\hline                  
 Support   & Increase Support  &  Increase Support \\
 Oppose    & Decrease Support  &  Decrease Support \\     
\hline                          
           &        Hawkish           & Dovish  \\
\hline
 Support   & Increase Support  &  Increase Support \\
 Oppose    & Decrease Support  &  Decrease Support \\
\hline
\end{tabular}
\caption{Summary of results consistent with extensive elite leadership of public opinion on military alliances. These predictions assume that isolationists and doves are disposed to oppose alliances, while hawks and internationalists are likely to support alliance participation. Predictions relative to the baseline average of alliance support within each disposition.}
\label{tab:arg-sum}
\end{center} 
\end{table}


%\begin{table}[hbt!]
%\begin{center}
%\begin{tabular}{lccc}
%                 Scenario  &                  & Hawkish  & Dovish  \\  
% \hline
%\multirow{2}{*}{Baseline} &  Internationalist & Support  &  Mixed   \\
%                         &   Isolationist     & Mixed  &  Oppose \\
%\hline                  
%\multirow{2}{*}{Elite Support} & Internationalist   & Increase Support  &  Increase Support \\
%                            & Isolationist    & Increase Support  &  Increase Support \\     
%\hline                          
%\multirow{2}{*}{Elite Oppose} & Internationalist   & Decrease Support  &  Decrease Support \\
%                            & Isolationist    & Decrease Support  &  Decrease Support \\
%\hline
%\end{tabular}
%\caption{Summary of results consistent with elite cues leading public opinion on military alliances. These predictions assume that isolationists are disposed to oppose alliances, while hawks are likely to support alliance participation. Predictions relative to the baseline average of alliance support within each disposition.}
%\label{tab:arg-sum}
%\end{center} 
%\end{table}




% formation vs maintenance
Before discussing the research design, there are two important considerations. 
First, alliance formation and maintenance are distinct processes \citep{Snyder1997}. 
Therefore, I consider alliance formation and maintenance in separate survey experiments to assess whether the public views making a new alliance commitment and upholding an existing treaty differently. 


% long-run cycles
Second, feedback between elite cues and public opinion is plausible in the long run. 
Perhaps public opinion shapes elite cues, which in turn alter public opinion. 
Elites could respond to growing alliance skepticism by encouraging opposition, or attempting to lead countervailing alliance support.
Such feedback takes time, and would be most obvious in the context of longstanding alliances.
This analysis can therefore establish part of a potential feedback cycle by identifying who responds to elite cues.  
If elites have limited or minimal influence, they may be more likely to follow public opinion.
I now describe how I assess the sources of alliance attitudes. 


\section{Research Design}


% justify conjoint: 
I use two conjoint survey experiments to unpack U.S. public support for forming and maintaining military alliances. 
Information about observed alliances bundles elite support and alliance characteristics. 
Conjoint experiments allow researchers to decompose such composite phenomena and compare multidimensional treatments \citep{Hainmuelleretal2014}. 
For example, \citet{BechtelScheve2013} assess how institutional design affects public approval for climate cooperation. 


% describe rating tasks
Both conjoint experiments ask individuals to rate and support participation in defensive military alliances with randomly generated profiles of alliance characteristics and elite cues. 
In the alliance formation experiment, I ask respondents about five hypothetical new alliances. 
The alliance maintenance experiment presents five hypothetical existing alliances.


In the experiments, I first measure key respondent characteristics.  
Individual pretreatment measures of partisanship, hawkishness and internationalism structure subgroup analyses examining how individual concerns shape baseline support for alliance participation and responses to the treatments. 
After measuring key individual factors, I present a hypothetical alliance with a randomly generated profile of elite cues and characteristics in a table.
Once respondents read the table, I ask them to rate the hypothetical alliance on a scale from 0 to 100 and express approval of alliance formation or maintenance with a yes/no question. 
I then present four more randomly generated alliance profiles, so each respondent rates five hypothetical alliances in a single-profile conjoint design.%\footnote{A two profile design would ask respondents to choose between two alliances, each with a random set of characteristics.} 


% Add a table with conjoint attributes. 
Each alliance partner profile is drawn from the attributes in \autoref{tab:conjoint-vars}.
Every attribute has multiple levels.
The full alliance profile selects one value from each attribute. 
The set of attributes and values captures theoretically interesting alliance characteristics and generates plausible profiles.\footnote{There are no restrictions on value combinations in the alliance profiles. I employ this uniform randomization because all of these alliance profiles are plausible. This also generates crucial variation in elite cues.}
I randomize attribute order at the respondent level, so the table of attributes is the consistent for each respondent. 
Drawing alliance profiles at random and providing multiple rating tasks in a conjoint experiment makes estimating the average marginal component effect (AMCE) for each alliance attribute straightforward \citep{Hainmuelleretal2014}. 


\begin{table}
\begin{adjustbox}{width = .99\textwidth}
\begin{tabular}{lc} 
\hline \\ 
\textbf{Attributes} & \textbf{Values} \\
\hline \\ 
Republican Senators & Support an alliance with this country. \\
                    & Oppose an alliance with this country. \\ 
                    
Democratic Senators & Support an alliance with this country. \\
                    & Oppose an alliance with this country. \\ 
                    
The Joint Chiefs of Staff & Support an alliance with this country. \\
                    & Oppose an alliance with this country. \\ 
                    
The Secretary of State & Supports an alliance with this country. \\
                    & Opposes an alliance with this country. \\ 
                    
Trade Ties          & The United States has minimal trade ties with this country. \\
                    & The United States has modest trade ties with this country. \\
                    & The United states has extensive trade ties with this country. \\ 
% modified from Tomz and Weeks 2013 APSR: https://web.stanford.edu/~tomz/pubs/TomzWeeks-2013-11-Appendix.pdf 
Partner Political Regime    & This country is not a democracy, and shows no sign of becoming a democracy. \\
                    & This country is a democracy, but shows signs that it may not remain a democracy. \\ % democ backsliding
                    & This country is a democracy, and shows every sign that it will remain a democracy. \\
                    
Partner Military Capability & 10,000 soldiers and spends 1\% of their GDP on the military. \\ % low
                    & 80,000 soldiers and spends 2\% of their GDP on the military. \\ % moderate
                    & 250,000 soldiers and spends 3\% of their GDP on the military. \\ % high 
                    
Shared Threat       & The United States and this country face minimal common threats. \\ 
                    & The United States and this country face modest common threats. \\
                    & The United States and this country face serious common threats. \\
                    
Recent Military Cooperation  & This country has not participated in recent U.S. military operations. \\ 
                    & This country recently supported U.S. airstrikes against terrorists. \\
                    & This country recently supported U.S. counterinsurgency operations. \\
                    & This country recently fought with the United States in a war. \\
                    
Financial Cost      & This alliance requires \$5 billion in annual U.S. defense spending.  \\ 
                    & This alliance requires \$10 billion in annual U.S. defense spending.  \\ 
                    & This alliance requires \$15 billion in annual U.S. defense spending.  \\ 
                    
Conditions on Support  & The alliance treaty promises military support in any conflict. \\ 
                    & The alliance treaty promises military support only if this country did not provoke the conflict. \\ 
                    & The alliance treaty promises military support only if the conflict takes place in this country's region. \\
                    
Defense Cooperation & None. \\ 
                    & The alliance treaty provides basing rights for U.S. troops. \\
                    & The alliance treaty includes a shared military command. \\
                    & The alliance treaty includes an international organization to coordinate defense policies.  \\ 
% Issue linkages                    
Related Cooperation & None. \\
                    & The alliance is linked to greater trade and investment with the United States. \\ 
                    & The alliance is linked to greater support for the United States in the United Nations. \\ 
                    
Region              & Europe. \\ 
                    & Africa. \\
                    & The Middle East. \\ 
                    & Asia. \\   
                    & The Americas. \\ 
                                                                            
\hline \\
\end{tabular}
\end{adjustbox}
\caption{Table of alliance attributes in conjoint experiment profiles. I use the same set of attributes as treatments in the alliance formation and maintenance experiments.} 
\label{tab:conjoint-vars}
\end{table}


% summarize table
The alliance profiles include many salient attributes.
Support or opposition from Republican and Democratic Senators, the Joint Chiefs of Staff, and the Secretary of State provide elite cues from elected officials, military leaders and diplomats. 
Independent random assignment of elite cues helps show which elites are most influential.


Elite cues in media reports often include other information besides elite cues \citep{BaumPotter2008}, so the experiments also present a series of alliance characteristics. 
I include key alliance characteristics such as trade ties \citep{Fordham2010}, regime type, shared threat, military capability \citep{Johnsonetal2015}, conditions on support, defense cooperation \citep{Morrow1994, LeedsAnac2005}, and issue linkages \citep{Poast2012}.
All of these factors shape the perceived value of an alliance. 
The regime type indicator includes nondemocracy, fragile democracy, and consolidated democracy, as individuals may believe that democracies should cooperate because they share common concerns and values \citep{Chuetal2021}. 
The financial costs reflect the most conservative association between an alliance commitment and U.S. military spending from \citet{AlleyFuhrmann2021}. 
Recent military cooperation provides an indicator of partner reputations \citep{Crescenzietal2012, GannonKent2020}.
I also randomize the region of the hypothetical alliance partner to mitigate confounding on other dimensions like cultural similarity.


The experiments use hypothetical alliances to generate generalizable results and randomly assign country and alliance characteristics. 
Meaningful experimental variation permits inferences about changes in elite cues and allied characteristics that are absent in many observed alliances. \footnote{While this raises potential confounding concerns, the regional indicator should help avoid confounding on other dimensions.}
Accounting for these alliance characteristics limits confounding on elite cues, as greater detail reduces that likelihood that any impact of elite cues is driven by inferred alliance characteristics.
It also mimics media presentations that bundle elite cues and information about an alliance to provides insight into what information shifts public attitudes. 
With fourteen unique alliance characteristics, the conjoint experiment provides a detailed picture of each alliance with more information than most media presentations.


% Justify number of attributes
Including fourteen attributes for each hypothetical alliance also ensures that attributes do not mask one another, but also that respondents are not overwhelmed and reduce the effort they put into assessing the full profile.
Studies of satisficing in conjoint experiments suggest that including fourteen attributes in a profile is unlikely to reduce data quality \citep{Bansaketal2019}. 
Furthermore, there is little evidence of satisficing when respondents are asked to rate or compare five profiles \citep{Bansaketal2018}.



\subsection{Sample and Individual Measures}


There are two experiments--- one for alliance formation and another for alliance maintenance. 
Each nationally representative sample contains 1,500 U.S. respondents, recruited through Lucid Theorem.
With an effective sample size of 7,500 from 1,500 respondents completing five rating tasks, the estimates will be under powered for very small effects, but should have enough power to pick up large differences and interactions. 


I measured key individual correlates of alliance attitudes for each respondent, focusing on partisan affiliation\footnote{I classified independent ``leaners'' as Democrats or Republicans, respectively. I coded pure independents or others that expressed no partisan lean as independents.} and foreign policy dispositions. 
I used standard questions to measure internationalism and militant assertiveness \citep{KertzerBrutger2016}.
Analyzing subgroups in conjoint experiments requires categorical measures of foreign policy dispositions and partisanship. 
To divide respondents into isolationists and internationalists, I coded agreement with the most common survey measure of isolationism as isolationism, and disagreement or a neutral stance as internationalism. 
The hawkishness index sums three questions about the use of force and war. 
Hawks scored above the midpoint of three on this scale, while doves scored three or lower. 
Finally, I interacted party affiliation, hawkishness and isolationism to analyze foreign policy dispositions within partisan groups.


To analyze the results, I first estimate unconditional average marginal component effects (AMCEs).
This establishes whether elite cues have influence in general. 
After that, I explore the extent of elite influence by examining how partisanship and foreign policy dispositions modify the impact of elite cues. 
To analyze alliance support in the partisan and dispositional subgroups, I estimate the overall mean choice for each group, then compare it to the marginal means of support under each attribute level.
Marginal means estimate average choices or ratings for each conjoint attribute level, averaging over all other treatments. 
I also employ omnibus F-tests to assess aggregate differences \citep{Leeperetal2020}. 


\section{Results} 


In these results, I first present the AMCEs of elite cues and alliance characteristics.
I then show how partisan identification, hawkishness and isolationism shape alliance attitudes. 
I find that elites exert extensive but incomplete leadership over alliance attitudes, but the consequences of elite cues depend on partisanship and foreign policy dispositions.
The subgroup analysis shows that in addition to wide variation in baseline alliance support, small subsets of both parties hold rigid opinions. 
\autoref{fig:joint-plot} presents the AMCE of elite cues and alliance characteristics on individual choices in the alliance formation and maintenance experiments.
Given the large number of factors, all results figures highlight the most salient AMCE estimates.\footnote{See the choice and rating AMCE figures in the appendix for a unified presentation of all the estimates. I include alliance characteristics to benchmark the relative weight of elite cues.}


The unconditional AMCE estimates suggest significant elite influence on alliance attitudes. 
Elite cues increase public support for alliance formation and maintenance. 
Support from Senators and the Joint Chiefs of Staff is especially influential.
Backing from the Secretary of State increases support for alliance formation and has a smaller positive effect on alliance maintenance choices. 


\begin{figure}
	\centering
		\includegraphics[width=0.95\textwidth]{../figures/joint-amce-plots.png}
	\caption{Average marginal component effect of elite cues and alliance characteristics on public support for forming or maintaining a hypothetical military alliance. Feature names in parentheses. Estimates with a dot at zero are the base attribute level. Components marked with abbreviated labels and some attributes omitted to make the plot more legible.}
	\label{fig:joint-plot}
\end{figure}


Some alliance characteristics influence alliance attitudes as well. 
Allied regime type is particularly consequential. 
Established democracy increases support for alliance formation and maintenance.
The magnitude of the established democracy AMCE is comparable to elite cues.   
Weak democracies are marginally more likely than non-democracies to receive public support.


Issue linkages also encourage support for alliance formation and maintenance. 
Linkages to trade and investment with the United States increase support for alliance participation, relative to an alliance with no linkages. 
Political issue linkages in the United Nations bolster individual support, though this effect is smaller than that of trade. 
This adds a public opinion mechanism to prior findings that issue linkages can facilitate new alliance agreements \citep{Poast2012} and bolster alliance credibility \citep{Poast2013}. 


Last, alliance context and costs shift public attitudes. 
Trade ties and serious common threat encourage support for alliance maintenance. 
Relative to the lowest annual military spending cost, annual costs of \$10 billion or more decrease support for upholding an alliance.  
Respondents also view alliances in Europe or the Americas more favorably than commitments in Africa. 


The above results assume that individuals respond in the same way to different cues and alliance characteristics. 
But individual concerns, especially the confluence of partisanship and foreign dispositions, structure foreign policy attitudes.
Examining how these factors change individual responses shows who elite cues lead.  



\subsection{Partisanship, Hawkishness, Isolationism, and Elite Cues}



In this section, I estimate support for alliances across respondents with different partisan affiliations and foreign policy dispositions.  
This analysis helps establish who elites lead in public opinion towards alliances. 
Extensive elite leadership implies that elite cues change attitudes regardless of foreign policy dispositions. 
In the following, I plot the marginal means of support for alliance participation for distinct foreign policy dispositions within the two major parties.  


\autoref{fig:party-dispo-form-el} and \autoref{fig:party-dispo-main-el} show the marginal means of support for alliance formation and maintenance from elite cues across partisan and foreign policy disposition subgroups.\footnote{See the appendix for details on the distribution of foreign policy dispositions across party identification.} 
Each panel plots the marginal mean of support for each elite cue within every categorical combination of militant assertiveness, internationalism, and partisanship.
In every panel, a solid vertical line marks a marginal mean of .5 and a dashed line summarizes the baseline or average alliance choice across all attributes and levels for that group.  
Both figures show how individuals in each group respond to elite cues. 


Co-partisan elite cues exert substantial influence on alliance attitudes.
Responses to elite cues depend on a complex combination of foreign policy dispositions and partisanship. 
The same foreign policy dispositions have distinct implications for alliance attitudes among Republicans and Democrats, so partisanship matters.
At the same time, foreign policy dispositions produce substantial differences in alliance attitudes within parties.  


\begin{figure}[htpb]
	\centering
		\includegraphics[width=0.95\textwidth]{../figures/party-dispo-form-el.png}
	\caption{Marginal means of support for forming hypothetical alliances across party identification and foreign policy dispositions given different elite cues. For each group, the estimates mark the marginal mean of support for alliance participation under different alliance treatments. The solid vertical line marks a marginal mean of .5, while the dashed line marks the average choice across all levels. Components marked with abbreviated labels to make the plot more legible. Independents omitted.}
	\label{fig:party-dispo-form-el}
\end{figure}


Among Democrats and Republicans, hawkishness increases support for alliance participation. 
There are partisan differences in this relationship, however, as hawkish Democrats express higher support for alliance participation than hawkish Republicans. 
Hawkish and isolationist Democrats are the strongest supporters of alliance participation. 
Hawkishness also increases support for alliance participation among isolationists. 


Isolationism alone does not reduce support for alliance participation.
Rather, alliance opposition comes from opponents of international engagement and military force. 
Isolationist and dovish individuals are the greatest skeptics of alliance formation and maintenance. 
Although Republican doves are rare, they are integral to alliance skepticism in the GOP. 
Dovish Democrats are also more likely to oppose alliance participation.  


\begin{figure}
	\centering
		\includegraphics[width=0.95\textwidth]{../figures/party-dispo-main-el.png}
	\caption{Marginal means of support for maintaining hypothetical alliances across party identification and foreign policy dispositions given different elite cues. For each group, the estimates mark the marginal mean of support for alliance participation under different alliance treatments. The solid vertical line marks a marginal mean of .5, while the dashed line marks the average choice across all levels. Components marked with abbreviated labels to make the plot more legible. Independents omitted.}
	\label{fig:party-dispo-main-el}
\end{figure}


In addition to shifting baseline alliance attitudes, foreign policy dispositions change individual responses to elite cues. 
Isolationists are less likely to heed elite cues. 
Internationalist Democrats respond strongly to support from Democratic Senators, and also look to cues from the Secretary of State and Joint Chiefs of Staff. 
Hawkish and isolationist Democrats express consistent support for forming and maintaining alliances, albeit with some attention to military elite cues. 
The strongest alliance supporters in the Democratic party thus hold rigid alliance attitudes.


Among Republicans, hawks are most receptive to elite cues. 
Regardless of their view of international engagement, there are clear differences in alliance support for hawkish Republicans based on Republican Senate support or opposition.
Hawkish Republicans also follow cues from military elites, and internationalist hawks in the GOP pay further attention to diplomatic elites. 
As a result, Republican elites can lead alliance attitudes among individuals who are disposed to support forceful international engagement and thereby constrain alliance support among the most likely alliance backers in their party. 
The gap in hawkish Republican attitudes from differences in Republican elite support is especially pronounced in the alliance formation experiment. 
Dovish and isolationist Republicans pay little attention to elite cues. 
In the reverse of the Democratic party, the most likely alliance supporters in the Republican party have plastic alliance attitudes. 


% group size
Rigid alliance attitudes are rare. 
In the alliance formation experiment, 7\% of Republicans and 25\% of Democrats hold foreign policy dispositions with rigid alliance attitudes. 
In the alliance maintenance experiment, 9\% of Republicans and 23\% of Democrats hold the same rigid set of alliance attitudes.
This makes a larger share of the Republican party responsive to elite cues.



% highlight differences in formation and maintenace
Individuals hold distinct attitudes towards alliance formation and maintenance. 
Forming new alliances has lower baseline support than maintaining existing treaties, so elite cues are crucial for new alliances. 
Only hawkish Democrats express clear support for alliance formation--- other respondents are divided or oppose new treaties.
Dovish isolationists dislike new alliances, though elites can persuade Democrats with this disposition. 
Whether elites support or oppose an alliance determines whether it has majority or minority support within each party. 


Alliance maintenance commands more robust support than alliance formation. 
Regardless of elite cues, the overall average and marginal means of support for alliance maintenance are almost all above .5. 
Even dovish isolationists in the GOP express a split verdict on alliance maintenance on average.
Although elite cues can change public attitudes, their direct influence on support for existing alliances has substantive limits.


% wrap up
These results suggest that elite cues exert extensive but somewhat limited influence on alliance attitudes.
Partisanship and foreign policy dispositions set the bases from which elite cues lead most of the public. 
There is an important partisan asymmetry in alliance attitudes as well. 
Democrat leaders can lead alliance skeptics and have less influence over the most committed alliance supporters. 
Republican elites can lead alliance supporters, but cannot persuade committed alliance skeptics. 
As a result, elite cues exert substantial influence on most individuals in both parties, but foreign policy dispositions restrain elite influence. 


An online appendix provides further support for these results. 
In the appendix, I examine marginal means by partisanship and foreign policy dispositions alone, analyze responses to an open-ended question, and compare results with the continuous rating measure of alliances to inferences from the choice question.
All these checks are consistent with these findings. 


\section{Discussion and Conclusion} 


% Overview
I find that elites often lead public alliance attitudes.  
Individuals follow cues from co-partisan elites, but their exact response depends on partisanship, hawkishness and isolationism. 
How alliance attitudes change depends on where they start.
Moreover, a few individuals hold rigid alliance attitudes. 
Allied democracy, issue linkages, shared threat, financial cost and trade have noticeable impacts as well.  


% thus, net on puzzle
Elites have extensive influence on public opinion towards alliances, but foreign policy dispositions constrain their influence. 
The most committed alliance supporters ---hawkish isolationist Democrats--- pay little attention to elite cues.
Similarly, elite cues have no impact on the most committed alliance skeptics --- dovish and isolationist Republicans. 
The result is a partisan gap in elite leadership of alliance attitudes. 
Republicans can lead co-partisan alliance supporters, while Democrats can lead co-partisan alliance skeptics. 


% bring it in on observed alliances
These findings have three implications for understanding public attitudes towards U.S. alliances like NATO. 
First, the Republican and Democratic parties contain committed alliance skeptics and supporters, respectively.
Roughly a quarter of Democrats are strong, almost unconditional alliance supporters, while roughly 8\% of Republicans are staunch alliance skeptics.
Outside these groups and independents, most Americans follow elite cues in forming alliance attitudes. 
This makes whether elites follow fixed alliance attitudes in their party a critical issue, because it could polarize the electorate.  


Second, my findings support the view that elite-driven public opinion cycles could make democratic commitments less reliable \citep{GartzkeGleditsch2004}. 
Although the results suggest that many members of the public hold considered opinions \citep{PageShapiro1992}, they also show substantial elite influence. 
Elite opposition rarely pushes alliance attitudes into majority opposition to existing treaties, but elite cues can reduce aggregate support in both major parties.
In the Republican Party, elite opposition creates an even split in alliance maintenance attitudes. 
Negative cues from military or diplomatic elites would bolster the impact of skeptical politicians and cut public support. 


% why NATO robust under Trump? 
Finally, these results help us understand public opinion towards alliances like NATO during the Trump administration.
Although Trump often criticized U.S. allies, alliance commitments usually commanded majority support throughout his administration \citep{PewNATO2020}. 
The relative stability of alliance attitudes reflects Democrats' aversion to Trump, allied democracy, countervailing cues from other elites and high baseline support for existing alliances. 
For many Republicans, hawkishness offsets isolationism in alliance attitudes.
Although Trump likely increased Republican skepticism of alliances, concern that he would inspire resurgent isolationism in the Republican party may have been overstated.
Isolationist and dovish Republicans did not change their alliance attitudes in response to Trump, as his rhetoric matched their views. 


% limitations
These findings have some limitations. 
For one, while the sheer variety of alliances means that the above profiles are plausible, generalizing from survey experiments is challenging. 
The artificial nature of a survey experiment provides essential control to disentangle public attitudes, but no hypothetical alliance can fully reflect real world commitments.
Some confounding of elite cues is possible, as the experiment cannot include every potentially relevant alliance characteristic. 
Moreover, elites have other ways to move public opinion besides direct cues.
As such, this may be an easy first test for the influence of elite cues on alliance attitudes. 


% can't show pandering
While this paper provides new insight into elite leadership of foreign policy opinion, it does not give a comprehensive account of the issue.
It shows that elites can lead, but not when and why they choose to exercise that influence. 
How much and when elites decide to follow fixed alliance attitudes in their party also falls outside the scope of this paper. 
Understanding the long-run dynamics of leading and following and when elites employ different strategies is a crucial subject for future research. 


% non-us results- might be different, subject for future inquiry
Furthermore, this study focuses on the United States, which has an unusual alliance network. 
Though public opinion towards alliances in the United States is important, attitudes in other countries matter as well. 
Future research should examine the sources of alliance attitudes in other countries. 


% future stuff: feedback, content of elite cues
These results provide a foundation for further inquiry into the domestic politics of military alliances. 
Two questions are especially interesting in this respect.
First, how much feedback takes place between public opinion and elite cues? 
When do politicians follow rigid alliance attitudes or lead in a competing direction? 
Politicians might view marginal opinion shifts due to threat or allied democracy changes as an opportunity to encourage or arrest further changes in public support.
Second, would leaders face significant public disapproval if they withdrew from an alliance? 
This study focused on generic support, but future research should build on \citet{TomzWeeks2021} and examine specific alliance policy changes. 


These questions address how elites form and maintain domestic coalitions around international engagement. 
In the 75 years since the end of World War II, shifting elite cues, partisanship, generational experiences and allied characteristics may mean different groups back alliances today than in 1950. 
Tracking changes in the domestic coalitions backing alliances is another worthwhile task for future research.


% wrap it up 
In conclusion, elite cues exert extensive influence on alliance attitudes, subject to some important limits.
Most individuals heed elite cues, but subsets of both major parties hold more rigid alliance attitudes. 
Alliance attitudes therefore reflect a complex mixture of elite cues and individual considerations. 



\newpage

% Bibliography
 
\bibliography{../../MasterBibliography} 




\end{document}
